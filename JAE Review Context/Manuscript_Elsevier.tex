%% ===================================================================
%% This is a draft of your manuscript converted to Elsevier's format.
%% ===================================================================

\documentclass[preprint,12pt,authoryear]{elsarticle}

%% ===================================================================
%% PREAMBLE
%% ===================================================================

%% Language and font encodings
\usepackage[english]{babel}
\usepackage[utf8]{inputenc}
\usepackage[T1]{fontenc}

%% --- Removed Packages ---
%% The 'elsarticle' class handles the following, so they are removed to avoid conflicts:
%% - geometry: Margins are set by the class option (e.g., 'preprint').
%% - setspace: Line spacing is handled by class options ('preprint' is double-spaced).
%% - titlesec: Section formatting is handled by the class.
%% - caption: Caption formatting is handled by the class.
%% - natbib: This is loaded automatically by 'elsarticle'.

%% Math & fonts (Compatible packages are kept)
\usepackage{amsmath}
\usepackage{amssymb}
\usepackage{times}

%% Graphics, Tables, and Misc Utilities (Compatible packages are kept)
\usepackage{graphicx}
\usepackage{verbatim}
\usepackage{pdflscape}
\usepackage{subcaption}
\usepackage{array}
\usepackage{ragged2e}
\usepackage{booktabs}
\usepackage{makecell}
\usepackage{multirow}
\usepackage{dcolumn}
\usepackage{longtable}
\usepackage{tabularx}
\usepackage{ltablex}
\usepackage{threeparttablex}
\usepackage{color,soul}
\usepackage{textcomp}
\usepackage{url}
\usepackage{tikz}

%% Bibliography Style - Use Elsevier's author-year style
\bibliographystyle{elsarticle-harv}
\setcitestyle{authoryear, open={(},close={)}}

%% Hyperlinks (load last)
\usepackage[colorlinks=true, allcolors=blue, urlcolor=blue, linktoc=all]{hyperref}

% Journal Name (for the footer)
\journal{Journal of Accounting and Economics}

\begin{document}

%% ===================================================================
%% FRONT MATTER
%% This section uses Elsevier's specific commands for title, authors, etc.
%% ===================================================================
\begin{frontmatter}

\title{Firms' Real and Reporting Responses to Taxation: A Review\tnoteref{t1}}

%% Title footnote
\tnotetext[t1]{We thank Michelle Hanlon and Wayne Guay for their editorial guidance. We also appreciate the detailed feedback from and discussions with Jennifer Blouin (JAE conference discussant), Matthias Breuer, Elisa Casi, John Core, Lisa De Simone, Michael Devereux (discussant), Scott Dyreng, John Graham (discussant), Shane Heitzman, Svea Holtmann, Jeff Hoopes, Martin Jacob, Eva Labro, Ed Maydew, and Johannes Voget. We are grateful for research assistance from Lauren Andrews, Jensen Ahokovi, Manon Francois, and Qihang Zhang. We further thank seminar and conference participants at the 2024 Journal of Accounting and Economics Conference, the 2024 UNC Tax Symposium, the 2024 EIASM Conference on Current Research in Taxation, the 2023 LBS-Stanford GSB Global Tax Conference, the 2023 Mannheim Taxation Conference, Brigham Young University, the University of Mannheim, and WU Vienna. Marcel Olbert acknowledges the financial support of the Research and Materials Development Fund (Olbert\_2023\_8934) at London Business School.}

%% Authors and Affiliations (using labels for linking)
\author[stanford]{Rebecca Lester\corref{cor1}}
\ead{rlester@stanford.edu}

\author[lbs]{Marcel Olbert}
\ead{molbert@london.edu}

%% Corresponding author text
\cortext[cor1]{Corresponding author}

%% Affiliations
\affiliation[stanford]{organization={Stanford Graduate School of Business},
            addressline={655 Knight Way}, 
            city={Stanford},
            state={CA},
            postcode={94305}, 
            country={USA}}

\affiliation[lbs]{organization={London Business School},
            addressline={Regent's Park}, 
            city={London},
            postcode={NW1 4SA}, 
            country={United Kingdom}}


\begin{abstract}
Taxation is a central economic policy tool, with governments increasingly using tax policy to stimulate local economic growth and also regulate multinational firms. We review the empirical literature that studies the effect of tax policies on firms' investment, employment, and other real outcomes. Building on the neoclassical theory of corporate taxes and tangible investment, we propose an organizing framework for our review that captures the wide set of tax policies and firm responses examined in accounting research. This framework highlights four dimensions along which accounting scholars contribute to the literature: i) documenting the role of financial reporting incentives as a moderating factor in firms' real responses, ii) studying firms' reporting versus real responses, iii) quantifying real effects of tax disclosure regulations, and iv) improving measurement of firms' tax status and proxies for investment and employment. We identify open questions for future research and suggest new international, federal, and local settings that may help uncover underlying mechanisms driving observed economic phenomena. Specifically, we encourage scholars to further distinguish firms’ reported and real responses to tax changes and improve measurement of these outcomes, especially in settings related to environmental taxation or settings in which tax avoidance and real outcomes are closely linked.
\end{abstract}

\begin{keyword}
%% keywords here, in the form: keyword \sep keyword
Tax policy \sep business taxation \sep real effects \sep tax disclosure \sep literature review
%% JEL codes here, in the form: \MSC code \sep code
\MSC[2020] H20 \sep H23 \sep H25 \sep H26 \sep H71
\end{keyword}

\end{frontmatter}

%% ===================================================================
%% MAIN TEXT
%% ===================================================================

\section{Introduction}
\label{sec:introduction}

Taxation is a first-order policy tool, with governments using tax policies to raise revenue for government spending, to redistribute wealth, and to change taxpayer behavior. Governments are increasingly using tax policy to target \textit{business} activities, such as investment spending, technological change, and the cross-border allocation of activity and income. 
Given the central economic implications of these policies for firms, regulators, and society, a large and growing body of research across accounting, economics, finance, and law studies the relationship between taxation and ``real'' firm outcomes, such as investment and employment. 
While the public economics literature has traditionally examined these topics, Figure \ref{fig:publications} shows that research in finance and accounting has grown substantially since 2010.

We review this recent empirical literature, focusing on accounting research that studies both firms' real and reporting responses to taxation.
Understanding both types of responses is important because they collectively determine the effectiveness of tax policies in inducing the intended response. 
For our review, we adopt \cite{leuz2016economics}'s definition of ``real effects'' as changes in firms' behavior that affect tangible transactions between the firm and its stakeholders in the real economy, such as investments and other types of resource allocation. Practically, this means that we review research examining investment- and employment-related measures as direct responses to taxation, as these two measures have been the predominant ``real'' outcomes studied by tax accounting scholars.  
We also consider studies that examine firms' financial or tax ``reporting'' responses in settings where tax policies were mainly expected to induce real effects.\footnote{Examples of reporting responses to taxation include firms reclassifying an expense as research and development (R\&D) to increase R\&D tax credit claims, even though the underlying economic transaction did not relate to R\&D, as well as firms responding to tax incentives with cross-border income shifting in lieu of real investment spending. In identifying reporting responses, we also follow \cite{hanlon_review_2010}, who distinguish between ``real'' and ``reporting'' effects. We acknowledge the difficulty in defining real effects and clearly distinguishing them from reporting responses. Firm decisions often produce a wide range of effects on stakeholders, and reporting responses can lead to follow-on real effects, such as those driven by capital market feedback or managerial learning \citep{bond2012real, roychowdhury2019effects}. This challenge is similar to the complexity of measuring earnings manipulation (e.g., \citealp{mcvay2006earnings,zang2012evidence}). We address this issue from a theoretical perspective in Section \ref{section:framework_extended} and discuss its implications for future empirical research in Section \ref{section:synthesisinvestment}. }
We exclude tax research that does not examine real or related reporting outcomes as the primary variable of interest and instead refer readers to other tax literature reviews.\footnote{For example, we exclude work solely focusing on the determinants of firms' tax avoidance \citep{wilde2018perspectives}, profit shifting (\citealp{heckemeyer_multinationals_2017} and \citealp{dharmapala_profit_2019}), accounting for income taxes and associated capital market outcomes \citep*{Graham2012}, debt financing \citep{HanlonHeitzman2022}, and effects of tax disclosure regimes beyond our focus on real effects (\citealp{muller2020determinants} and \citealp*{hoopes2022taxdisclosure}).} 

\begin{figure}[h!]    
\centering
\caption{Research on the Real Effects of Taxation in Top-Tier Academic Journals}
\label{fig:publications}

   \includegraphics[width=0.95\linewidth]{tax_realeffect_bar_topjournal_count_byyear.png}
\subcaption*{\textit{Notes:}  This figure presents the number of papers on firms’ real responses to taxation, published in top-tier accounting, finance, and economics journals from 2000 to 2023.\footnotemark }
\end{figure}
\footnotetext{To generate these statistics, we scraped journal websites and extracted paper abstracts and titles from all published issues.  We then applied a large language model classification algorithm to identify those focused on the real effects of taxation.
Specifically, we used a four-shot chain-of-thought prompting approach with OpenAI’s ChatGPT 4o-mini model. We first provided the model with our definition of real effects, specified the relevant outcome variables, and then supplied four manually classified examples along the dimensions of (i) tax vs. non-tax and (ii) real effects vs. no real effects. The total number of classified papers ($N$) across the top 11 journals in the three disciplines is 272. Our review also includes a broader sample of papers, including relevant work from other journals and unpublished papers, as defined in Footnote \ref{scope_papers}. Figure \ref{fig:publications_alltax} presents the statistics for all papers classified as tax research, encompassing all topics and outcomes, including those beyond the scope of real effects ($N$=894). }

The main goals of our review are to i) synthesize and highlight key findings and policy implications in the growing body of research, ii) critically evaluate the state of the literature, pointing out findings that appear inconsistent with theoretical foundations or topical areas where application of recent advances in research design would improve causal inference; and iii) suggest open questions for future research in accounting that have the potential to impact policy and practice.


To frame our review, we start with a summary of the canonical theory in \cite{hall1967tax}. Their neoclassical model predicts a negative effect of higher corporate income taxes on firms' marginal investment in tangible assets through changes in the firm's cost of capital. As this theory has limitations and simplified assumptions, we also describe how other theories and frameworks such as the one by \cite{scholeswolfson1992} augment the theoretical foundation for tax accounting research. This extended discussion is important because accounting research often examines a broader set of outcomes, tax policies, and firm frictions than those modeled in \citet{hall1967tax}.

The remaining sections of our review are organized around the set of firm outcomes examined by accounting scholars. For example, we start with a review of research on investment outcomes, including capital expenditures, innovation, and M\&A, as well as employment outcomes, such as employment levels and compensation.  We then discuss the research examining \textit{where} firms locate investment and employment and how this research intersects with the large literature on the ``reporting'' outcome of profit shifting.  We follow this section by reviewing research on other manifestations of real outcomes, including risk-taking, indirect real effects, and investment targeted at corporate sustainability.  

To identify the papers included in the review, we obtained all tax papers published in leading academic accounting journals since  \cite{hanlon_review_2010}'s call for more research on the real effects of taxation.
We also conducted a targeted search for studies on firms' real responses to tax policy in leading finance and economics journals.\footnote{\label{scope_papers} Specifically, we searched for studies on taxation based on the titles, keywords, and abstracts in the following journals: \textit{Journal of Accounting and Economics, Journal of Accounting Research, The Accounting Review, Management Science, Contemporary Accounting Research,
and the Review of Accounting Studies}. %This exercise resulted in a baseline sample of \textit{500 papers}. 
We also systematically searched for articles in the following leading finance and economics journals, as well as relevant field journals:  \textit{The Journal of Finance,
Journal of Financial Economics,
The Review of Financial Studies,
Journal of Corporate Finance,
Review of Finance,
American Economic Review,
Quarterly Journal of Economics,
Journal of Political Economy,
Review of Economic Studies,
Econometrica, 
Accounting, Organizations and Society,
European Accounting Review,
Journal of the American Taxation Association,
Journal of Public Economics,
National Tax Journal, American Economic Journal: Economic Policy,  and the
Journal of International Economics}.
Where appropriate, our review also mentions work published in accounting journals before 2010 or unpublished working papers that have the potential to make significant contributions to the literature.} 
By including papers that examine a broad range of tax policies and a host of real outcomes, as well as papers at the intersection of real and reporting responses, we augment the literature included in the earlier review by \cite{jacob2022review}. 

In the course of reviewing the literature, we identified four key contributions of tax accounting scholars to the research on real effects.  
First, accounting research documents that real responses to tax policy vary based on financial reporting incentives (e.g., \citealp*{williams2021real, goldman2022fasb}).
Second, accounting research shows that firms engage in reporting responses  instead of real spending in response to  tax incentives (e.g., \citealp*{Coles2022}). 
Third, accounting research studies real effects of mandatory  tax-related disclosure regimes, documenting often unintended real responses (e.g.,  \citealp*{Rauter2020,jacobwentland2022real}). 
Fourth, accounting scholars improve the
measurement of firms' responses to tax incentives depending on their tax status and jurisdictional exposure (e.g., \citealp*{dyreng2009using,Bethmann2018,blouin2023double}).  We highlight these contributions throughout the review when discussing specific papers or topics, and we also discuss opportunities to advance our contributions in each category.

Our review of the literature also identified several open research questions, which we  discuss within each sub-section pertaining to a particular real outcome, as well as in dedicated ``synthesis'' sections. As one example, we highlight several instances in which the literature has either  overlooked important questions -- like the dynamics of firms' investment responses -- or produced seemingly contradictory findings.\footnote{For example, several papers suggest a substantial capital expenditure (capex) investment response to tax loss carryback rules (e.g., \citealp{Dobridge2021,Bethmann2018}), but other work shows a limited capex response by loss firms (e.g., \citealp{edgerton2010investment}).  As another example, prior work on foreign direct investment finds less investment abroad in response to a change to a territorial tax regime from a worldwide tax regime \citep{Arena2015, Amberger2021}, but in the same setting, \cite{liu2020wheredoes} finds increased foreign investment.  Focusing on country-by-country disclosure by European firms, \cite{desimone2022realeffects} find that firms alter their foreign investment abroad, but other work finds little to no effect among U.S. companies (e.g., \citealp*{nessa2023effect}).} As another example, the literature is often unclear about whether policies motivate new, incremental investment spending or instead only subsidize investment that would have occurred regardless of the incentive.  This lack of clarity is in part a function of the use of differing measures and quantification approaches, which we detail in accompanying tables summarizing and comparing empirical measures and data sources. As a third example, when discussing the literature at the intersection of real location decisions and tax planning (focusing specifically on tax-motivated cross-border profit shifting), we identify a ``chicken-egg'' problem:  i) Do firms first choose where to locate and then exploit income shifting strategies on top, or ii) do firms first shift income and then substantiate that activity with increased investment and employment in a low-tax jurisdiction, or iii) do firms engage in these activities together (simultaneous tax planning and real responses)? Determining when and under which circumstances investment (the egg) or income shifting (the chicken) comes first is very challenging, but future research disentangling these responses directly informs policymakers aiming to reduce tax avoidance and cross-border tax competition. Finally, as a fourth example, we discuss specific ways in which accounting researchers -- who, admittedly, are not climate scientists -- can contribute to the rapidly growing research on environmental and sustainability policies. A key path for our field is to focus on reporting responses to environmental taxes and related disclosure rules, as well as on measurement of financial-related metrics.  These metrics include firms' investment in green technologies, tax payments associated with changes in emissions and environmental regulations, and subsequent financial performance.

Beyond these types of open questions, we also highlight three broader opportunities for future research. These opportunities span several different types of firm responses. First, empirical work thus far lacks an overarching theory that integrates insights from both public economics theory (on the relation between investment and taxation) and accounting disclosure theory (on the role of measurement and disclosure in firm real effects, e.g.,  \citealp{kanodia2007real,kanodia2016real}).
Filling this gap is important for two reasons.  The first, as highlighted in \cite{kanodia2007real}, relates to measurement: the way that accountants measure and report transactions can have real effects on firms' decisions.  By extension, measurement of real responses to tax policies hinges on measurement of outcomes and constructs that are based on the accounting system.  Understanding the imprecision introduced by the accounting system is important for both measuring the policy effect on a firm (i.e. the ``treatment intensity'' of a policy) as well as the measurement of the outcome variables of interest.  Second, disclosure can have both direct and indirect effects, but disentangling these effects and evaluating the relative costs and benefits in empirical studies can be difficult.\footnote{For example, the direct effect of a disclosure regime such as country-by-country reporting is to provide more information to tax authorities. An indirect effect is to increase the cost of locating in low-tax jurisdictions, as increased disclosure potentially increases both the likelihood of tax enforcement actions and reputational costs.} 

Disclosure theory also provides insights about the impact of regulatory regimes on earnings quality. Tax-induced reporting changes can alter the quality of firms' reporting information, but little is known about the effects of these tax-induced reporting changes on firm investment, governance, and capital market outcomes. Understanding these effects is timely given the ongoing and increasingly heated debate about whether to tie firms' tax treatment to financial reporting income \citep*{Graham2014,Hanlon2021financialaccountingfromtaxreforms,gaertner2025investor}.

%Opportunity \#2: Bi-lateral information flows with other fields.  
A second area of opportunity is further incorporating insights from other areas of accounting research as well as from finance and economics. Such interdisciplinary work can produce impactful contributions to practice and policy, especially in cases where well-informed decisions rely on unique insights from accounting academics. One key example is the case when tax law affects financial reporting information \citep{Hanlon2021financialaccountingfromtaxreforms}. Accounting researchers have already made notable strides in areas such as measuring tax status, corporate financing, tax compliance, or investor responses to firms' tax affairs (e.g., \citealp{Blouin2010, graham2017tax, Dyreng2017, slemrod2017does, edwards2020capital, gaertner2025investor}), potentially inspired by earlier calls for interdisciplinary approaches \citep{maydew2001empirical, hanlon_review_2010}.\footnote{Notable examples advancing this direction, such as \citet{jacob2019consumption}, are highlighted and discussed throughout our review.} To further advance interdisciplinary work, we offer several concrete suggestions.
First, future accounting research can better integrate insights from other fields by, for example, building on and referencing public economics theories that often provide direct predictions for relationships examined in our research.\footnote{For example, \citet{Huang2020} examine the effectiveness of R\&D tax incentives in stimulating innovation spending by firms, a question of central relevance across disciplines. Given the strong interest in this topic among economists \citep{akcigit2022taxation}, accounting research addressing similar questions can have the greatest impact when engaging with the literature and active scholars outside of accounting.} 
Second, accounting researchers can leverage their competitive advantage in measuring firms’ exposure to setting-specific tax treatments to address research questions that are central to other fields such as debt financing or asset pricing (e.g., \citealp{Blouin2010,gomezolbert2023}). 
Third, we encourage researchers to discuss and relate empirical results to economic magnitudes found in related studies from other disciplines.
Finally, successful interdisciplinary research should adopt data sources and empirical designs that reflect best practices in causal identification, following advances from the methodological literature that is relevant across fields. At the same time, accounting scholars can help shape the literature by documenting factors often overlooked in other disciplines -- such as reporting incentives and disclosure regimes -- that impact policy effectiveness, while at the same time ensuring accurate measurement of the central constructs of interest used by researchers across all three fields.
In practice, embracing an interdisciplinary approach requires increased interactions between accounting scholars and those in other fields. This includes participating in non-accounting conferences and workshops, seeking feedback from and offering feedback to scholars working in related fields on specific projects, collaborating with scholars from other fields (e.g., through joint research, teaching, and dissemination of insights), and also seeking publication opportunities in non-accounting journals or policy outlets.

%Split for length
Recent research on  the U.S. Tax Cuts and Jobs Act (TCJA) exemplifies the opportunities for cross-disciplinary learning across fields. 
Accounting researchers were among the first to provide empirical evidence about the TCJA, including measuring firms' exposure to the law, quantifying investor responses, and documenting financial statement and reporting implications (e.g., \citealp*{gaertner2020making,Dyreng2023effectustax,lynch2023earnings}).
More recent work in both accounting and economics reveals that real responses to this tax law occur along many margins. Research examining these distinct features contributes to the literature because conclusive evidence on a big and policy-relevant topic such as the TCJA typically emerges from a collection of academic studies. However, making such a contribution requires a good understanding of existing and concurrent work across fields to articulate the unique and novel finding.  Therefore, future accounting research should not only build on existing work in our field, but also on studies such as \cite{chodorow2023tax} by (for example) qualitatively and quantitatively comparing how inferences and magnitudes across the fields relate to each other.  Furthermore, we need more evidence such as that in \cite*{kelley2023just}, which documents substantial reporting (reclassification) responses, a margin often overlooked in other disciplines but equally informative to policymaking.      

%Split for length
A third area of opportunity for future accounting research is to provide new evidence about the underlying mechanisms driving real responses. Specifically, we hope that research will move past documenting whether investment and employment effects occur to further providing causal evidence about the mechanisms and dynamics driving these responses.  For example, an increasing number of international policy developments, including global minimum taxes, destination-based taxes, and carbon border adjustment mechanisms (CBAM), use a multilateral approach to address cross-border competition. As many of these involve new reporting and disclosure requirements, they offer novel settings for accounting scholars.  Understanding how and to what extent these reporting and disclosure requirements affect multinational investment is important to explain heterogeneity in the policy response and also inform future coordinated policy efforts. As another example, tax authorities increasingly turn to disclosure regimes as the primary manner to pursue improved ``transparency.'' However, to the extent that disclosure regimes are ineffectively or partially implemented, disclosure may be an ineffective policy tool. Pursuing an interdisciplinary approach to studying firms' responses to these developments -- and also explaining the mechanisms driving these responses -- will further increase the relevance of tax accounting research.

Studying real effects of taxation is a promising path for accounting scholars. 
As emphasized throughout the synthesis sections of our review, we do not merely call for \textit{more} studies examining \textit{more} firm outcomes in tax settings that have already been previously studied. Instead, we highlight the unique contributions accounting researchers can make to policy-relevant knowledge by focusing specifically on the intersection of real and reporting responses along the four dimensions outlined above. In particular, we note a surprising lack of evidence along the first two dimensions: understanding how reporting incentives  alter real responses and examining how firms adjust their reporting in lieu of or alongside real responses. We look forward to such future work that advances our understanding of these real effects and the role that accounting information, firm reporting, and disclosure play in these outcomes.

We organize the rest of our review as follows. In Section \ref{section:theory}, we discuss the theoretical foundations for why taxation should affect real business outcomes, and we propose the organizing framework for the manuscript. Sections \ref{section:investment} and \ref{section:investment_types} review the empirical work on real and reporting responses. Each subsection covers a particular type of real outcome; for example, Sections \ref{section:firmsreal_investment} and \ref{section:employment} discuss the two traditional measures of real activity, investment and employment, respectively, whereas Sections \ref{section:risktaking} and \ref{section:sustainability} focus on outcomes studied in more recent work, such as risk-taking and responses to environmental taxes, respectively.  Section \ref{section:conclusion} concludes with some general discussions that relate to several aspects of the work we reviewed.

\section{Theory and Conceptual Framework} \label{section:theory}

\subsection{Basic Neoclassical Theory of Taxes and Corporate Investment}\label{section:framework_basic}
The canonical theory in \cite{hall1967tax} models firms’ marginal investment in \textit{depreciable physical capital} in response to changes in the \textit{deductibility of investment costs} given existing \textit{corporate income tax rates}.  Eq. \ref{equ_theory_coc_maindraft} presents the key insight that firms invest as long as the marginal revenue product of capital ($MPK$) equals  the user cost of capital ($CoC$):

\begin{equation}\label{equ_theory_coc_maindraft}
f'(I)=MPK=r\frac{{(1-\tau_{c}z)}}{(1-\tau_{c})}=CoC
\end{equation}

Specifically, greater tax deductibility of investment in fixed tangible capital ($z$) increases marginal corporate investment ($I$) by reducing the after-tax user cost of capital and thereby lowering the hurdle rate for investment ($\downarrow$ $MPK$).\footnote{The concept of the user cost of capital in public economics differs from the capital market-based concept in accounting and finance. The after-tax user cost of capital describes the cost of using one unit of capital for one period, accounting for the opportunity cost of investing the available funds elsewhere at the real interest rate, the loss in value of the capital due to economic depreciation (or renting assets to obtain capital services), and the tax benefits associated with capital expenditures.} If firms’ initial investment costs are not fully tax deductible ($z<1$), as is typically the case in many jurisdictions, the model also shows that higher corporate income tax rates ($\tau_{c}$) discourage investment. Figure \ref{fig:framework_basic}  visualizes this, and Section \ref{oa:theory} of the Online Appendix provides  a detailed discussion. 

\begin{figure}[h!]
\centering
\caption{Basic Neoclassical Theory of Taxes and Corporate Investment}\label{fig:framework_basic}
   \includegraphics[width=0.7\linewidth]{simple_framework.png}
        \subcaption*{\textit{Notes:} This figure illustrates the basic theory on the relationship between corporate tax incentives and firms' investment as modeled in \cite{hall1967tax}.
        }
\end{figure}

\subsection{Limitations of Basic Theory and Implications for Empirical Research}\label{section:framework_limitations}
\cite{hall1967tax} is a useful starting point for research on the real effects of taxes because the general view is that corporate tax changes affect the economy primarily through firms’ investment decisions. Consequently, a vast number of papers on firms’ real responses to tax incentives implicitly or explicitly build on this model. 

However, the basic theory has limitations due to its simplifying assumptions of the neoclassical investment framework. Importantly, the neoclassical theory assumes that managers maximize the net present value (NPV) of after-tax cash flows in the absence of agency conflicts and other potential frictions.\footnote{Among other factors, the theory relies on a Cobb-Douglas production function with constant returns to scale in a static framework with constant responses from the representative firm. It also assumes a competitive economy with price-taking firms, no adjustment costs, and no uncertainty. Recent work has extended the basic model to account for financial and agency frictions, policy uncertainty, variation in tax incidence, and general equilibrium effects at the macroeconomic level, among other factors. For instance, \cite{zwick2017tax} show that investment responses can be larger than the standard theory predicts if investment inputs are long-lived. \cite{alvarez1998tax} provide a theoretical model of tax rate policy uncertainty and corporate investment. \cite{mumtaz2018policy} discuss and show aggregate negative investment effects of tax policy uncertainty. As our goal is to offer a brief summary as the foundation for the empirical work we review, we do not offer a complete model incorporating all possible tax policies, firm responses, and underlying frictions. Instead, we highlight that this work is important for scholars to consider when applying theory to motivate empirical predictions.} However, these assumptions are not met in many empirical settings.  For example, early work in accounting shows that incorporating these factors helps explain why firms do not always respond to tax incentives as the neoclassical investment model would predict \citep{scholes1990tax, scholes1992firms}. These findings led to \cite{scholeswolfson1992}'s proposed framework of ``all taxes, all parties, and all costs'' that allows for these other factors when studying firm responses to taxation. 

Prior research has predominantly used the \cite{scholeswolfson1992} framework to explain differences in and determinants of corporate tax avoidance (e.g., \citealp{Schwab2022}). We emphasize that research on firms’ investment and resource allocation decisions can also adopt this approach to produce impactful evidence. Specifically, we recommend framing inferences in the context of standard theory (e.g., \citealp{hall1967tax}) and incorporating frameworks that extend basic assumptions, such as those in \citet{scholeswolfson1992} or similar theories.\footnote{For instance, \cite{domar1953case} and \cite{zwick2017tax} provide theories showing that the effects of tax savings on investment should be more pronounced among financially constrained and smaller firms in which cash tax savings are particularly valuable.} One example is managers maximizing private benefits rather than the highest after-tax NPV projects (e.g., \citealp{edwards2016trapped,hanlon2014effect}). Another example is managers facing uncertainty and limited awareness about different tax incentives, which may impede their ability to correctly estimate the monetary value of these incentives \citep{graham2017tax}. A final example is firms’ tax incidence and bargaining power vis-a-vis multiple stakeholders, which ultimately affect the extent of firms’ responses to changes in tax burdens and the margins of those responses \citep*{winter2024incidence,dyreng2022tax}. In short, researchers should consider and discuss deviations from theory to inform predictions and interpret empirical results.
 
\subsection{Framework for the Review and Suggestions for Future Research}\label{section:framework_extended}
Empirical research in accounting and related fields has examined firms’ responses to taxation beyond the primary relationship depicted above in Figure \ref{fig:framework_basic}. Moreover, we anticipate that future research will expand the set of firm decisions and tax policies under investigation. Our review covers this wide range of research, and Figure \ref{fig:framework_extended} presents the framework we used to organize our discussion. The left side of the figure shows the range of \textit{outcomes} examined (i.e., the ``y'' variables in a reduced form regression), and the right side includes different types of \textit{tax policies} studied (the ``x'' variables).  Below, we discuss five notable parts of this organizing framework.

\begin{figure}[ht!]
\centering
      \caption{Framework for the Literature Review: Tax Policies and Firm Responses} \label{fig:framework_extended}
   \includegraphics[width=1\linewidth]{conceptual_framework_v1.png}
    \subcaption*{\textit{Notes:} This figure illustrates the framework for our literature review.
        }
\end{figure}

First, the right side of the figure includes the two tax parameters in the original model (the corporate income tax rate and depreciation of tangible capital investment), as well as other tax system features examined in the accounting literature. Examples of these other tax system features include incentives that reduce the tax base, such as repatriation tax holidays, innovation box regimes, or the U.S. Domestic Production Activities Deduction (``DPAD'').  By lowering a firm's taxable income, these incentives effectively reduce the corporate income tax rate, thereby impacting real responses.\footnote{For example, firms participating in the 2004 U.S. repatriation tax holiday, as passed in the American Jobs Creation Act, received an 85\% dividends-received-deduction (``DRD'') on amounts repatriated from the foreign subsidiary to the U.S. parent. This DRD effectively lowered the tax rate on repatriated funds to 5.25\% (85\% DRD x 35\% U.S. statutory income tax rate). The U.S. DPAD and some innovation box regimes operate in a similar manner by permitting reductions that lower the tax base and thus effectively also lower the tax rate $\tau_{c}$.} Our review also discusses changes in non-income taxes, such as sales or value-added taxes.    
 
Second, in addition to including specific types of tax policies, the right side of the figure includes tax enforcement and tax disclosure rules.  Tax enforcement operates in a similarly indirect manner as changes in tax bases: greater tax enforcement (indirectly) increases expected tax burdens (e.g., \citealp{allingham1972income,sandmo1974note}).\footnote{\cite{allingham1972income} and \cite{sandmo1974note} apply Becker's theory of crime to  individual tax evasion, extending the underlying mechanisms from these theories  to the business setting; specifically, increasing the penalties for tax evasion and/or the probability of detection should similarly reduce business tax evasion.} Thus, if firms expect higher tax burdens due to greater enforcement, investment will decrease as in the standard case of a tax rate increase or a lower tax depreciation deduction. However,  tax enforcement effects are nuanced, because enforcement reflects tax authorities' actions on behalf of governments  and can thus be seen as a form of political risk \citep{hassan2019firm,gallemore2024tax}.
Greater tax enforcement can also \textit{increase} tax certainty and thus lower tax risk. In this case, firms might respond by increasing investment  
(e.g., \citealp{Pindyck1988,bloom2009impact,bloom2014fluctuations}). Thus, the ex-ante theoretical relation between enforcement and real outcomes is unclear.   

Tax disclosure regimes are distinct from the other policies included in the figure because they do not directly alter the treatment of business transactions under tax law. Thus, researchers cannot simply invoke the standard theory when motivating studies examining the real effects of disclosure. However, changes in mandated or voluntary reporting directly impact the amount or type of firm information provided to external stakeholders, including shareholders, creditors, consumers, employees, citizens, the press, and tax authorities. As managers anticipate that stakeholders will use the disclosed information in ways that can affect the firm, firms may alter their tax planning and real decisions.\footnote{ \cite*{muller2020determinants} and
\cite{hoopes2022taxdisclosure} provide  recent reviews of the literature on private or public tax disclosures. Most studies reviewed in \cite{hoopes2022taxdisclosure} focus on the relationship between these disclosure rules and firms' information environment, with a particular focus on the effectiveness of tax disclosure rules in limiting tax avoidance behavior. This effect typically occurs through the revelation of incremental information about a firm's tax avoidance activities to tax authorities.}  


Third, the left side of Figure \ref{fig:framework_extended} lists a number of outcomes beyond capital investment.  These include other types of investment studied in the prior literature, most notably mergers and acquisitions (M\&A) and innovation-related investment (R\&D) for which the original theory has been adapted (e.g., \citealp{Rao2016}). The left side also includes other outcomes, such as those motivated by alternative theoretical frameworks (i.e., risk-taking based on \citealp{Domar1944}), and those indirectly impacted by tax policy changes (i.e., employment). When reviewing these outcomes, we discuss the relevant theoretical extensions important to motivate the predicted effects.\footnote{For example, there is only an indirect effect of tax policy changes on employment; the extent to which labor responds is a function of the complementarity or substitutability with investment. 
 Employment effects also depend on tax incidence, which is the extent to which different parties ultimately bear the burden of tax. If tax increases reduce a firm's demand for capital, they could also potentially decrease the demand for labor. Depending on the business tax incidence, the firm's responses might also lead to changes in factor prices (i.e., wages, input goods, final goods) \citep{goolsbee1998investment}, such that employees, suppliers, or customers bear some of this tax burden \citep{harberger1962incidence,Clausing2013,suarez2016benefits}.}    

 Fourth, the left side also includes reporting outcomes.  While tax policies are often intended to motivate real outcomes, such as increasing investment and jobs, these responses can be costly.  Firms may instead engage in other ``reporting'' responses first, trading off the benefits from claiming the incentive with the relative cost of each response. Some of the earliest work in accounting focuses on trade-offs of incentives \citep{scholes1992firms,maydew1997tax}, often documenting that reporting responses dominate. \cite{Slemrod1992} formalizes this framework when thinking about real responses, outlining that firms first alter the timing of transactions, then recharacterize transactions with accounting choices, and finally engage in the real response. Furthermore, to the extent that financial reporting quality is impacted because of firms' tax-induced reporting changes, then tax policies potentially have broader capital market and real effects \citep*{Hanlon2021financialaccountingfromtaxreforms,roychowdhury2019effects}. Understanding all of these reporting effects is particularly important given that imprecise reporting can also impact real activity \citep{kanodia2007real,zang2012evidence}.
 
Finally, the arrows for ``Moderating Factors'' highlight that there is substantial variation in the extent to which tax policies impact firms' real outcomes. Research studying such moderating factors typically departs from the basic assumptions of \cite{hall1967tax}, which model a single firm maximizing the NPV of each investment in the absence of non-tax frictions such as reporting incentives, agency conflicts, and uncertainty. The “all taxes, all costs, all parties” framework from \citet{scholeswolfson1992} offers a helpful lens for examining moderating factors that are prevalent in many accounting research settings.  One such example is firms' reporting incentives. The intuition is that firms face public reporting costs (or benefits) when responding to tax policies. As a consequence, these reporting costs, which include the disclosure of proprietary information, the reduction of financial reporting income, or a decrease in the usefulness of accounting information for financial statement users, impede the real response.\footnote{See \cite{Hanlon2021financialaccountingfromtaxreforms} for a  discussion and \cite*{Graham2012} a review of the literature.}

Beyond documenting the moderator factor related specifically to financial reporting costs, accounting researchers also apply their institutional knowledge to identify important variations in firm responses and settings in which the standard theory assumptions do not hold.  One example of the latter is documenting the presence of agency issues that impact firm investment decisions  (e.g., \citealp*{edwards2016trapped,Hanlon2015repatriation}). Another example is showing that managers do not always optimize on tax policy parameters that maximize NPV projects (e.g., \citealp*{graham2017tax}). As a final example, while tax reforms are typically widely recognized, decision-makers may perceive the impact and certainty of a given policy change differently for their business. Accounting researchers’ traditional focus on information and stakeholder perceptions provides valuable tools to measure these moderating factors and explain differential or unexpected firm responses to taxation (e.g., \citealp{gallemore2024tax}).

A comparison of Figures \ref{fig:framework_basic} and \ref{fig:framework_extended} demonstrates the broad application of the  \cite{hall1967tax} model and highlights the expanded set of policies and outcomes examined in recent empirical work. Throughout our review, we refer back to this framework to organize our discussion and highlight key contributions of accounting scholars.

\section{Investment Levels and Location in Response to Taxation}\label{section:investment} 
\subsection{Corporate Investment and Taxation}\label{section:firmsreal_investment}
\subsubsection{Capital Expenditures (Capex)}\label{section:capex} 
\paragraph{Theoretical Underpinnings and Overview}
We start with capital expenditures (``capex'') because investment in tangible depreciable assets is most directly predicted by the theory discussed in Section \ref{section:theory}. Further, it is the most commonly studied real outcome among accounting scholars.  

The economics field provides foundational evidence, focusing primarily on documenting and quantifying the negative relation between capex and tax rates across a number of settings. Estimates of the elasticity of investment with respect to the tax-adjusted cost of capital range from -0.25 to -1.0 \citep{hassett2002tax,house2008temporary,deMooij2008,jacob2022review}. Recent work refines these estimates through improved identification \citep{zwick2017tax} and  documents heterogeneous responses due to adjustment costs and policy uncertainty \citep{cooper2006nature, chen2022tax, Guceri2021,QipingZwick2024}. 

Accounting scholars have contributed to this literature in three primary ways, building on the Scholes-Wolfson framework.  This literature i) further quantifies the investment response to tax changes, ii) documents reporting and real responses (variation based on ``all costs''), and iii) studies responses to non-income taxes (``all taxes'') and shareholder-level taxes (``all parties''). 

\paragraph{Quantifying the Investment Response to Realized Tax Savings}
\cite{Green2022}, \cite{Dobridge2021}, \cite{Guenther2020}, and \cite{Bethmann2018} quantify how much of cash-tax-related savings is actually spent on capex. They find that firms invest between \$0.33-\$0.40 per dollar of tax savings, with estimates varying by firm and across macro-economic conditions.  While these papers help document the extent of firm' investment response, the open question is whether this investment is efficient.  Are firms using the cash tax savings to invest in either positive NPV projects or instead inefficient projects that should remain unfunded from a welfare standpoint?  Furthermore, does the tax-motivated investment drive economic growth, or does the investment reflect agency-driven managerial decisions, with limited positive spillover benefits for other firms and the local economy? 

\cite{Bethmann2018} make the most progress towards answering these questions. The paper exploits tax loss carryback rules across countries. Tax loss carrybacks provide firms that incur a tax loss the ability to net the loss against profits from a prior year to receive a refund of earlier taxes paid. The study documents a \texteuro{0.33} increase in spending per \texteuro{1.00} of tax savings, but it concludes that carrybacks stimulate inefficient investment based on observing that the investment occurs among the more distressed but least productive firms. The use of exogenous variation and a well-motivated and rigorous fixed effects structure supports these inferences. 

However, the investment efficiency question remains open, as other research that also uses the tax-loss-carryback setting arrives at the opposite conclusion \citep{Dobridge2021}. To what extent can the difference in efficiency outcomes stem from different settings (U.S. vs. Europe), periods (crisis vs. non-crisis), or firm types (public vs. private)?  Another possible explanation for the different effects can be the literature's relatively coarse measures of capex, given challenges in both observing discrete capex decisions and measuring the corresponding long-run performance. Further, both \cite{Bethmann2018} and \cite{Dobridge2021} focus on tax loss policies, leaving it unclear to what extent the investment responses to other tax changes are efficient. To progress the literature, future work should go a step further and examine whether tax-motivated investment specifically contributes to firm- and economy-wide growth, ideally incorporating both agency theories and techniques that best enable measurement of spillover effects.

\paragraph{Examination of Reporting and Real Responses}

A central finding from the early tax accounting literature is that firms respond to tax policies with reporting changes \citep*{scholeswolfson1992}. These studies primarily focus on reporting responses related to tax avoidance, such as changes in reporting of pre-tax income. However, reporting responses should also arise when considering ``all costs" that influence firms’ investment choices, as reported investment affects both the tax base and firm disclosures to external stakeholders. Specifically, if transaction or adjustment costs are large, firms may first respond to tax incentives by shifting the timing of, or the accounting for, transactions in lieu of engaging in the actual intended ``real'' investment \citep{scholes1992firms,Slemrod1992}.  

Despite having a well-developed framework for understanding when reporting responses occur, our literature provides surprising little empirical evidence about these choices when studying investment-related outcomes. To our knowledge, only two papers explicitly examine and quantify these differential responses. \cite{Lester2019} uses a difference-in-difference design around the implementation of the Domestic Production Activities Deduction, which \textit{lowered} firms' tax burdens, to separate the real investment and reporting responses.  She finds that real spending was delayed for several years because firms first engaged in inter-temporal and cross-border income shifting, consistent with \cite{Slemrod1992}. \cite{Coles2022} employs the ``bunching" methodology developed by \cite{saez2010taxpayers} to quantify the decline in firms' taxable income following corporate tax rate \textit{increases}.
%\footnote{The increases could be due to statutory rate changes or simply a firm progressing through the progressive tax structure as it earns more income.} 
The central finding is consistent across both papers, even though they examine different firm samples and settings: reporting responses to tax changes are substantial. For example, \cite{Coles2022} document that such responses represent two-thirds of the total response among small private firms.

Future research could build a more comprehensive understanding of reporting responses, as these responses -- when combined with real responses -- determine the overall effects of tax policies and inform cost-benefit evaluations. Furthermore, there is extremely little evidence on the manner in which firms actually execute these reporting responses. The literature speculates that firms accomplish these reporting transactions using a variety of strategies, ranging from delaying invoices to income shifting. Understanding these types of ``tax-earnings management" techniques (i.e., managing taxable income in response to tax incentives) is important for two reasons: first, to better understand responses aimed primarily at "gaming" tax benefits, and second, to identify how these tax reporting choices affect other firm decisions and stakeholders. For example, some transactions will affect reported financial statement income, which in turn has both equity capital market effects and debt capital market effects.  The latter can occur to the extent reporting shifts impact the firm's ability to meet existing debt covenants, as suggested by \cite{maydew1997tax}.

Convincing evidence in this area requires a deep understanding of institutional details and accurate measurement of both reported and real investment outcomes. Given accounting scholars' expertise in understanding the economics behind firms' reported numbers, researchers are well-positioned to provide valuable insights into the full costs and benefits of investment-related tax policies and their broader impact on firm decision-making.


\paragraph{Capex and Non-Corporate Income Tax Policies}\label{section:firmsreal_investment_nonincometax}
%Review
While the accounting literature traditionally focuses on corporate income taxes, several studies examine the impact of other taxes, including consumption taxes, dividend taxes, and employees' personal income taxes. This work highlights the importance of considering ``all taxes'' and ``all parties'' when measuring the business response to tax policies \citep{scholeswolfson1992}. The key finding is that while these other types of taxes do influence firm investment choices, they operate through different mechanisms.

Most evidence covers the role of taxes on dividends. The theoretical link between dividend taxes and firm investment is based on a firm's marginal source of investment financing. To the extent that the marginal source of investment financing is external equity, then higher dividend taxes increase firms' cost of capital; consequently, firms issue less external equity, which then constrains investment.  This ``old'' (or ``traditional'') view implies that reductions in dividend taxes should increase firm investment.  In contrast, the ``new'' (or ``neutrality'') theory states that firms finance investment with internal capital. Under this theory, dividend taxes should only affect the \textit{efficiency} of investment.  This is because, after a reduction in dividend tax rates, shareholders -- whose capital is ``locked in'' to a firm under relatively higher dividend rates -- will re-allocate their capital from over-investing firms to financially constrained, under-investing companies \citep{chetty2010dividend}. The theory predicts that, in this case, both over- and under-investment problems are alleviated, leading to increased investment efficiency.

Several accounting and economics papers examine these predictions, but the empirical evidence is mixed about whether dividend taxes impact the level or the efficiency of firm investment (\cite{yagan2015capital, Becker2013, Chay2023}). The most recent evidence indeed points to increased investment efficiency, using the large 2003 U.S. dividend tax cut as the empirical setting \citep{Chay2023}. The paper identifies affected firms -- and captures treatment intensity -- based on those firms for which the marginal investor is an individual subject to dividend taxes (following \cite{blouin2017measuring}).  
The paper finds sizeable improvements in investment efficiency, with over- (under-) investing firms' efficiency increasing by 4.5\% (0.6\%) of total assets. Furthermore, the paper finds that the overall level of investment is unchanged, consistent with earlier findings in \cite{yagan2015capital}.

The open question is whether these effects also apply to share repurchases, the other predominant and rapidly growing form of payout. To the extent that taxes on share repurchases impedes shareholders from re-allocating capital across firms, we should observe similar effects as with dividend taxes.  However, because shareholders can voluntarily decide whether to participate in a share repurchase program, and because tax rates on repurchases may be lower than dividend tax rates (such as prior to 2003 in the U.S.), it is unclear if the new-view theory also holds for this form of payout. Policy changes, such as the excise tax passed by the Inflation Reduction Act of 2022, provide recent settings to evaluate this question and, more specifically, to assess the extent to which firms respond by substituting between dividends and repurchases.   

The literature also examines consumption taxes paid by consumers \citep{jacob2019consumption} and personal income taxes paid by employees \citep{jacob2022pit}. In these studies, the theoretical link between these taxes and firm investment is primarily related to tax incidence.  For example, if demand for a firm's products is relatively elastic, then firms bear more of the consumption tax \citep{jacob2019consumption}. Similarly, firms may bear some portion of employees' personal income tax \citep{jacob2022pit}. In both cases, taxes lower a firm's financial performance, which in turn reduces firm investment.

Despite these well-executed studies, significant gaps remain. First, researchers must carefully evaluate the theoretical mechanism underlying why a given non-corporate income tax change affects investment. If the mechanism primarily relates to tax incidence, studies should clarify what ``new'' economic phenomena can be uncovered, given that the extant literature already provides substantial evidence on tax incidence effects. Second, we have even less evidence as compared to the income tax literature as to whether non-income taxes induce reporting adjustments or resource re-allocation in lieu of real investment activity.  Third, it remains unclear to what extent existing findings generalize across economies and tax types, particularly where consumption taxes are less salient than the European VAT and in which labor laws and markets differ.  

Understanding these issues in the context of non-income taxes is important given the recent decline in corporate tax revenues.  This decline, due in part to the erosion of countries' income tax base, the passage of corporate income tax reductions, and the rise of non-corporate entities, means that governments may increasingly shift to alternative non-income tax sources.  To provide relevant insights, future research on the real and reporting effects should consider these non-income and indirect taxes, particularly along the three dimensions highlighted above.


\subsubsection{Innovation }\label{section:firmsreal_innovation} %[Section by: BECKY]
\paragraph{Theoretical Underpinnings and Overview}

%Intro
Researchers define innovation as the improvement or invention of a production process, product, method, or platform \citep{arrow1972economic,romer1990endogenous,glaeser2024review}. As innovation is the key driver of economic growth in modern economies and because innovation outcomes are uncertain \textit{ex-ante}, governments use regulation (such as tax policies) to encourage private-sector investment. 

%Tax policy overview and theory 
Three types of tax policies affect innovation activities.  First, ``front end'' incentives, such as special deductions and R\&D tax credits, are tied to firms' expenditures on innovation activities. Second, ``back end'' incentives, such as innovation box (``IB'') regimes, provide lower tax rates on income from successful innovation \citep{Merrill2016, evers2015intellectual}. Third, general tax policies, such as low statutory tax rates on all business income, also affect innovation spending.  

The theory linking taxation to innovation investment extends \cite{hall1967tax}; see, for example, \citealp{Rao2016}). With respect to R\&D deductions, immediately deductible R\&D ($z=1$) motivates firms to innovate, similar in spirit to how bonus depreciation motivates tangible capital investment.\footnote{While wages to research personnel are typically tax deductible absent special tax policies, R\&D rules often extend this deductibility to tangible capital and other long-term investment outlays. Fully deductible R\&D costs for tangible capital ($z=1$) should minimize the impact of statutory tax rates, as shown in \cite{hall1967tax} for capex.} With respect to R\&D credits, the dollar-for-dollar tax benefit directly lowers the $CoC$ via cash tax refunds, thus increasing the marginal return to investment. ``Back-end'' incentives reduce tax burdens by changing the tax base through increased or super-deductions for innovation ($z>1$), by providing a lower rate on innovative income ($\tau_{c}$), or both. 

A large literature spanning economics, finance, and accounting studies the effectiveness of tax policies for innovation. The economics and accounting literature generally finds relatively large innovation effects of these tax policies, with elasticities in excess of one \citep{akcigit2022taxation, Rao2016, Guceri2019, agrawal2020taxcredits, berger1993explicit, klassen2004cross, Finley2015rdtaxcredit}. Accounting scholars further progress this literature by focusing on the inherently unique and opaque nature of innovation, which contributes to information asymmetry between the firm and its stakeholders (including both shareholders and tax authorities). Thus, accounting research examines the real effects of innovation tax policies through the lens of studying informational benefits and costs associated with claiming tax incentives. Specifically, accounting scholars (i) document non-cash-tax benefits (and costs) related to public disclosure of the R\&D tax credit; (ii) show heterogeneity in R\&D credit claims based on internal information necessary to substantiate the tax benefit; and (iii) study the role of all parties (shareholders) and all taxes (state taxes; capital gains taxes) in the investment choice.  In this section, we predominantly discuss work on ``input'' related incentives -- i.e., the R\&D tax credit.  There is also a substantial literature in accounting that focuses on ``output''-related incentives, most notably innovation box regimes. Because this literature examines multiple outcomes, including innovation \citep{bradley2015cross}, capex \citep{chen2023effect}, M\&A \citep{Bradley2021}, and employment \citep{bornemann2023effect}, and because these papers focus on cross-border effects of this policy, we discuss the policy extensively in Section \ref{section:allocation}. 


\paragraph{Benefits and Costs of Public Disclosure of R\&D Credits}
%Reporting incentives and responses
Two papers document that firms obtain non-tax benefits from public disclosure of the R\&D tax credit.  \citet{Hepfer2025} find that pre-IPO firms claim R\&D tax credits to signal the credibility of their innovation, thereby reducing information asymmetry with potential investors at the IPO. \citet{williams2021real} show that after the introduction of FIN 48, which delays recognition of R\&D tax credit benefits, firms reduce their investment in innovation.  They find a rather large 9\% reduction in innovation in response to the FIN 48 disclosure regime.

These papers suggest that R\&D credits have informational benefits beyond cash tax reductions, but questions remain about the exact mechanisms driving these benefits and their plausible magnitudes compared to earlier documented elasticities.
The papers point to two different mechanisms: \citet{Hepfer2025} argue that disclosing R\&D credit claims reduces investors' adverse selection concerns about unobservable R\&D investments, while \citet{williams2021real} suggest that myopic managers cut R\&D when the financial statement benefit declines. Subsequent studies in this area alternatively propose that FIN 48 introduces a cost related to increased tax authority scrutiny, leading firms to reduce marginal innovation that may not qualify for credits \citep{goldman2022fasb,goldman2023noninnovative}.
More evidence is needed to both quantify and explain what exactly drives the benefits that firms appear to obtain from public disclosure of innovation tax incentives. One suggestion for future research is to use designs that isolate or compare alternative mechanisms, both to identify the dominant mechanism and to also determine how big these effects are as compared to the cash tax benefits from the policy itself.  Such approach could possibly shed light on why and how much disclosure matters for firm's innovation decisions.

\paragraph{Internal Information Frictions Related to Innovation Tax Incentives}
 
Accounting scholars also document frictions that impede firms from claiming R\&D credits. Beyond the actual complex calculation of the credit, taxpayers also have uncertainty about whether certain expenses will qualify. On this basis, firms report R\&D credits as a common ``Uncertain Tax Position'' to the IRS \citep{Towery2017}, and increased IRS scrutiny is associated with both reduced innovation investment and R\&D credit claims \citep{cowx2022investment}. 

The open accounting question in this literature is to what extent internal R\&D information -- including the amount, quality, and location (centralized vs. dispersed) of innovative activities -- enables firms to accurately measure, track, and ultimately substantiate the credit claimed \citep{Huang2020, cowx2022investment}. For example, \cite{cowx2022investment} show that, when faced with increased IRS scrutiny, firms with relatively poor internal information quality report lower R\&D credits and invest less in innovation.  The key insight is showing that internal information frictions impede real investment activity as intended by the R\&D tax credit. While a good step forward, the literature lacks more evidence on the role of these information frictions for other benefits and whether they are also present for other types of investment (and if not, why). Furthermore, the literature would benefit from continued refinement of IIQ measures that best conform to the setting at hand.\footnote{As one example, in addition to using the original IIQ measures developed by Gallemore and Labro, \cite{cowx2022investment} also uses \cite{Samuel2022}'s more recent measure based on government contracting. The premise of using this measure is that government contractors require similarly granular details as that required for the R\&D tax credit.}  While data limitations certainly present a challenge, the growing trend of academics partnering with companies for field experiments may offer opportunities to ``crack open'' the black box of IIQ.

Beyond information frictions, open questions remain about other frictions that preclude many businesses from claiming R\&D tax benefits, including taxable status requirements and compliance costs.  First, many R\&D tax incentives, such as the R\&D tax credit, only convey current cash tax savings to firms with positive taxable income, excluding  some of the most innovative businesses  from receiving  benefits. The 2015 U.S. Protecting Americans from Tax Hikes (PATH) Act, which allows small companies to use the R\&D credit to offset \textit{payroll} taxes, offers a setting to study both the real and reporting responses of R\&D tax incentives among pre-revenue and pre-positive income firms. A key question is understanding whether these incentives increase survival rates of high-quality firms or  prolong the life of low-quality firms that should  exit the market.  

Second, there is little evidence on the compliance costs tied to certain innovation tax benefits. Unlike depreciation and bonus depreciation incentives, which are relatively straightforward to claim, R\&D tax credits are complex to obtain, as evidenced by practitioner guidance showing that companies hire large accounting firms for these calculations. However,  smaller tax advisory firms increasingly use AI tools to more effectively calculate firm-specific credits, raising questions about how  this shift in advisory practice impacts the number and types of companies claiming the credit and how firms spend their new-found R\&D credits.  Moreover, how does the internal information environment of these relatively small, often private companies impact the amount claimed and the ability of these firms to sustain the credit upon audit scrutiny?  Understanding these shifts would provide valuable insights into the compliance costs of innovation tax benefits.



\paragraph{The Role of ``All Parties'' and ``All Taxes'' in Innovation Investment}

Beyond national-level tax policies, the literature studies subnational policies targeted at innovation activities. 
The key finding is that higher state tax rates discourage innovation \citep{mukherjee2017corporate, glaeser2022proximity}. \cite*{Li2021} go beyond state tax rates, studying both reporting and real responses to state ``add back'' rules. These state tax rules limited certain R\&D deductions, with the primary goal of reducing the shifting of intangible-related income across state borders.\footnote{The rules require firms to add back the deductions for cross-border payments when calculating state-level taxable income, thereby eliminating the tax benefit associated with a reduced tax base.} Not only does the paper confirm that firms reduced the shifting of state income (a ``reporting'' effect), but the paper also documents a significant decline in patenting activity (a ``real'' effect).  Thus, the key finding is showing that firms' real activity responds to reporting policies as well as to tax rates and incentives. 

\cite{he2022} examine how shareholder-level taxes affect firm investment. As with non-corporate income taxes discussed in Section \ref{section:firmsreal_investment_nonincometax}, the link between these shareholder-level taxes and firm investment occurs through a different channel than the traditional public economics theory. Specifically, the paper invokes managerial myopia, as in \cite{williams2021real}: myopic managers under-invest in innovation to optimize short-term valuation and appease short-horizon investors. The paper shows that favorable long-term capital gains taxes (or high short-term capital gains taxes) reduce myopia by motivating shareholders to retain capital in the firm for a longer period. The paper explicitly discusses the greater ``information asymmetry'' between managers and shareholders for innovation investment when explaining the mechanism for the observed effects. However, to best understand these information asymmetry issues, the literature needs more evidence that documents when these shareholder-level taxes matter, and why innovation is distinct or unique from other investment categories.

\paragraph{Additional Open Questions}
Specific to studying innovation outcomes, the tax accounting literature lacks evidence along three key  dimensions. First, we have surprising little evidence on when firms respond with reporting versus real responses.  Most evidence comes from  economics, showing  that firms shift and relabel activities in response to R\&D tax credits \citep{wilson2009beggar,chen2021notching}. Understanding which firms engage in this behavior and to what extent this occurs in other settings would help quantify the full firm responses to innovation incentives. 
Further, while the accounting literature identifies factors that hinder strong responses to tax incentives, it has not examined the specific dynamics or potential long-term responses to tax credits. Understanding this is especially important in the innovation setting, where R\&D investments are risky and long-term, and firms may delay their response to a tax policy change until uncertainty is resolved or internal frictions are addressed. 

Beyond understanding the reporting responses, there is little evidence measuring the nature or extent to which these tax policies achieve the intended spillover effects on the broader economy. On one hand, innovation policies may have positive spillover effects if firms mimic peers' innovation spending  \citep{kim2021innovation}. On the other hand, these policies may induce tax competition that results in firms simply shifting the location of innovation rather than engaging in new, incremental R\&D activity \citep{wilson2009beggar}. In this latter case, taxpayer dollars are used to subsidize activity that would have occurred regardless of the incentive, suggesting an inefficient use of government funds. Research that quantifies and evaluates the extent to which innovation tax policies stimulate local area activity is critical for evaluating the efficacy of these policies, but admittedly is challenging given that even identifying firm-specific effects can be difficult.  

Finally, the literature has largely ignored the role of other types of R\&D incentives, such as tax base reductions \citep{jacob2022review}. However, tax base incentives, such as deductions for R\&D, potentially convey much larger benefits given that they are not subject to the same restrictions for incremental investment as R\&D credits; for example, the U.S. tax deduction for R\&D is estimated to provide twice the benefits of the credit (IRS SOI 2024). Changes in tax bases, whether through reductions (such as the U.S. change in 2022) or increases (the Chinese super deduction) provide new settings for scholars to quantify how the removal of tax policies impacts real innovative spending (i.e., \cite{cowx2025174}). Beyond measuring the real impact of this change, new financial statement disclosures that accompany these changes provide a window into firms' innovation activities, presenting opportunities to contribute to other disclosure-related accounting research on innovation (i.e., \citealp{kohreeb}). 

\subsubsection{Mergers \& Acquisitions} \label{section:MA}%[Section by: MARCEL]

\paragraph{Theoretical Underpinnings and Overview}
Mergers \& acquisitions (``M\&A'') activity increases the size of the firm through the acquisition of a wide array of firm assets, including both tangible and intangible assets. The acquisition of tangible assets in an M\&A transaction follows the \cite{hall1967tax} theory: higher taxes affect the return earned on firm assets (i.e., income) and should reduce M\&A activity from the buyer's perspective. This theory should also extend to the acquisition of intangible and other assets, as the investment outlay is typically not immediately tax deductible and can lead to future tax depreciation and amortization only in the case of certain types of M\&A structures. However, the theoretical literature is less clear in this regard, with little actual work on this type of investment spending.  Despite this lack of theory,  the economics, finance, and accounting literature generally finds a negative relation between taxes and M\&A activity, consistent with this after-tax cost of capital view.  For example, \cite*{Arulampalam2019} uses country-level variation in tax rates to quantify a tax rate elasticity of acquisitions of -0.3 to -2.3, and \cite{huizinga2009international} find that international double taxation reduces the likelihood of foreign acquisitions. Work by accounting scholars similarly demonstrates this negative relation (i.e., \citealp{blouin2021, Bradley2021}).\footnote{This work predominantly uses tax changes to identify the effect of tax on investment. In constrast, \cite*{Chow2016} examine how firm-specific tax avoidance affects M\&A outcomes, finding that target firms attract higher takeover premiums if they disclose that they did not engage in tax shelters. This suggests that, while tax incentives motivate greater investment, consistent with theory, aggressive tax avoidance inhibits acquisition due to the potential costs imposed.}

We review how the accounting literature has contributed in three ways to the literature on M\&A: i) documenting the role of ``all parties'' including sellers, buyers, and investors in transactions; (ii) studying the efficiency of tax-motivated investment; and (iii) documenting trade-offs between financial reporting incentives and cash tax savings.  In Section \ref{section:allocation}, when reviewing the work on the geographic allocation of investment and employment, we also discuss a fourth contribution via studying the effects of international tax regimes. 

\paragraph{The Role of Buyers and Sellers}
The accounting literature demonstrates the importance of ``all parties'' when studying M\&A -- providing evidence that incorporating all parties' incentives is critical for explaining heterogeneity in transaction prices and structure.  Understanding the role of individual-level taxes in particular is critical for M\&A because the tax treatment of a deal is a function of both entity- and owner-level taxes.  

Earlier work examines under which conditions shareholder-level taxes impact M\&A activity, documenting that capital gains tax burdens inhibit efficient M\&A deals \citep*{ayers2004effect,todtenhaupt2020taxing}. The key link between capital gains taxes and investment activity is the ``lock in'' effect: when capital gains taxes are higher, target companies' shareholders are less willing to sell because they will incur a relatively higher tax.  More recent work advances this literature by further demonstrating that it is not all shareholders, but rather the CEO's tax liabilities specifically, that matter in these transactions, influencing both acquisition structure and deal price \citep{Hanlon2021ceo}.  

This finding raises a number of open questions for future research.  If only executives' tax liabilities matter, and if managers and shareholder incentives are misaligned, then do the number, amount, and types of deals reflect agency-driven investment choices by the CEO?  Do firms actually forego acquisitions in response to taxes, or instead, are deals likely to proceed, but with different deal characteristics related to structure or price that reflect executives' preferences?  Finally, it is unclear to what extent these findings translate to private companies, in which agency issues should be less pronounced and tax matters should play a relatively greater role. Understanding this is critical given that vast majority of firms and deals occur in the private market.  

\paragraph{Investment Efficiency and Deal Synergies}
One critique of studying the M\&A response to tax policies is that factors other than taxes likely dominate the M\&A decision, particularly with regard to  large deals.  However, one benefit of studying M\&A is that, unlike capex and innovation, it is relatively more straightforward to examine investment efficiency to understand whether tax policies achieve their intended goals or motivate firms to inefficiently overinvest. This is because researchers can observe discrete M\&A transactions with set announcement dates. Much of the literature uses capital market returns around the deal announcement date, interpreting market returns as a ``sufficient statistic'' for whether the deal is value-increasing to shareholders.

\citet*{blouin2021} is a good example of work that goes beyond documenting M\&A investment responses to examine investment efficiency.  The paper does so with three approaches: (i) testing acquirer shareholder market response, (ii) assessing synergies using combined cumulative abnormal returns (CAR) for both acquirer and target, and (iii) analyzing long-run tax benefits from the tax policy (DPAD) based on the target's activity.  Future work that can similarly incorporate multiple tests of investment efficiency will also help the literature move beyond documenting the effect to evaluating the type and nature of such investment.

\paragraph{Trade-offs between Financial Reporting Incentives and Cash Tax Savings}

The accounting literature has little evidence on the trade-offs that firms make between financial reporting incentives and cash tax savings, even though the M\&A setting presents several opportunities to study these trade-offs given differences in financial reporting incentives and the tax treatment of a transaction.  One notable exception is \cite*{lynch2019trade}, who use novel private company M\&A data to document book/tax trade-offs in the context of M\&A purchase price allocations. The key insight from the paper is documenting managerial discretion in purchase price allocation:  the paper finds that some firms allocate acquisition purchase price to optimize cash tax savings (via depreciation deductions) at the expense of intangible asset presentation for shareholders.\footnote{While both financial accounting and tax rules require that purchase price be  allocated by the acquirer across purchased assets according to the assets' fair market value, the study shows that managers exercise discretion. Specifically, managers of firms with greater cash tax incentives allocate greater amounts of the purchase price to fixed assets, permitting firms to recover their investment via tax depreciation quicker than if the purchase price were allocated to longer-lived intangible assets. However, because the purchase price allocation must be the same for tax and financial reporting purposes, the discretionary allocation means that firms understate the book value of intangible assets shown to financial statement users.}   

Beyond this paper, we have little understanding of these types of trade-offs that firms make and when tax or financial reporting concerns dominate in the context of M\&A investment. More evidence on the key value drivers, such as synergies or specific target-firm assets and liabilities with different tax and reporting implications, would enhance our understanding of how taxes influence different types of investment and the reporting incentives behind tax-driven M\&As.


%%%%%%%%%%%%%%%%%%%%%%%%%%%%%%%%%%%%%%%%%%%%%%%%%%%%%%%%%%%%%%%%%%%%%%%%%%%%%%%%%%%%%%%%%%%%%%%%%%%%%%%%%%%%%%%%%%%%%
%SYNTHESIS AND SUGGESTIONS FOR FUTURE RESEARCH
%%%%%%%%%%%%%%%%%%%%%%%%%%%%%%%%%%%%%%%%%%%%%%%%%%%%%%%%%%%%%%%%%%%%%%%%%%%%%%%%%%%%%%%%%%%%%%%%%%%%%%%%%%%%%%%%%%%%%

\subsubsection{Synthesis and Suggestions for Future Research}\label{section:synthesisinvestment}
%Intro
Beyond the specific challenges and opportunities for future research discussed in each subsection, we identify three key areas where future research can make valuable contributions to understanding firms’ investment responses to various tax incentives: (i) studying the dynamics of firm investment, (ii) considering adjustment along multiple investment margins, and (iii) improving measurement of key constructs. 

On the first point, there is surprisingly little direct evidence on the timing of firms’ investment responses to changes in tax incentives, despite theoretical and empirical evidence suggesting that firm-level investment depends on perceived uncertainty. Furthermore, investment is often lumpy and carries significant adjustment costs  (e.g., \citealp*{caballero1999explaining, bloom2007uncertainty, chen2022tax}). %These factors influence the dynamics of investment responses to changes in both tax and non-tax environments. Conceptually, many studies across fields remain agnostic about the specific periods in which firms respond and also to what extent.  % also in economics. also economics? %\cite{zwick2017tax} add examples of macro papers who are interested in dynamics at more aggregate level : cloyne2023short, their graphs can be use for presentation, need to read hennesy papers
However, the standard approach of comparing investment before and after tax changes may not capture these characteristics of investment, possibly even biasing estimates to the extent firms anticipate the changes (particularly if there is a long lag before implementation) or experience delays in their responses \citep{hennessy2020beyond, gallemore2024tax}.  This concern is especially important given the wide range of ``post period'' length often used in studies. There is significant potential for future research to model and depict the dynamics of these effects more accurately, especially in settings where firms’ tax policy expectations and adjustment costs vary widely. Understanding these dynamics are particularly important when effects may be explained by the interplay between firms’ reporting and real responses \citep{Lester2019}.\footnote{From a pure econometric perspective, results can be biased when tax policy and real activity are intertemporally related, invalidating the strict exogeneity assumption for firm-level fixed effects designs \citep{millimet2023fixed, breuerdehaan2024using}. We suggest that future research on investment responses to tax policy should clearly articulate the expected dynamics, use dynamic panel estimators or select a fixed effects design that avoids bias from intertemporal correlation \citep{Bai2016}, and, crucially, map out regression results over the event period to assess and discuss dynamics.} 
Furthermore, accounting researchers can leverage their expertise in firm-level qualitative disclosures to better investigate firms’ expectations around taxation and the timing of their investment strategies. Such measures, potentially generated through narrative approaches \citep{romer2010macroeconomic}, surveys, or textual analysis using large language models \citep{gallemore2024tax}, can also help isolate exogenous variation in tax policy across firms and over time, improving identification.

With respect to the selection of the investment outcome, future research should more comprehensively capture firms’ investment responses and evaluate the relative magnitude of effects across different types of investment. While studies often focus on a single type of investment, firms are likely to allocate tax savings across various categories. We do not suggest testing every investment type in every study, but recommend that scholars carefully consider theoretical predictions and clearly articulate why they focus on a particular outcome (or multiple outcomes) relevant to their research question.\footnote{Similar to \cite{hanlon_review_2010}’s discussion on tax avoidance measures, researchers should thoughtfully select an investment measure rather than applying multiple measures without a clear rationale. Studying multiple real outcomes for every tax policy change is not always appropriate, as some policies target specific types of spending. Additionally, using the same policy variation to examine different outcomes (i.e., multiple hypothesis testing) can lead to biased, inflated effects \citep{heath2023reusing}. That said, it is likely that firms adjust on multiple margins, which should be considered when motivating particular outcome(s).} When studying multiple measures, researchers should clearly explain why.  Examples of economic rationales for selecting multiple measures include assessing spillover effects, studying the complementarity or substitutability of production factors, quantifying co-location benefits, or documenting unintended policy consequences.

%Measurement \#1: accurate measurement
%\hl{RR: revise to be more specific and integrate Jennifer's discussion comment, also measurement tables}
Third, regarding measurement, future research should address two key issues: ensuring accurate measurement of the three investment types (capex, innovation, and M\&A), and considering incremental investment measures. Measurement issues are prevalent in business taxation research, often due to inconsistencies in data availability, variations in data fields across different datasets, and misunderstandings of what the data truly capture (e.g., \citealp{blouin2023double}). While the use of diverse datasets is natural -— especially with growing access to administrative data and new methods for tracking investment and employment -— this variety can make it difficult to compare findings across studies. Different papers often use varying measures for the same construct, even within the same context, leading to challenges in drawing consistent conclusions. Key measurement issues for capex, innovation, and M\&A need careful consideration, and researchers should aim to better align these measures to improve comparability across studies.\footnote{For example, commonly used data sources such as Compustat and Orbis produce very different measures of capex due to coverage differences (public vs. mostly private firms), geographic reporting (worldwide vs. country-specific), and unit of measurement (capex vs. total fixed assets). Even more problematic issues are present for innovation, which commonly uses three types of measures: input (R\&D expense, which most firms do not disclose), output (patents and trademarks, which ignores trade secrets that could represent the most innovative technologies for a firm), and intangible assets from the balance sheet (which are measured inconsistently across U.S. GAAP and IFRS and largely capture only \textit{acquired} assets. Finally, most work on M\&A focuses on the likelihood and number of deals, with fewer transactions having requisite price or deal premia information necessary to understand the economic magnitudes of transactions.} For further details on these measurement choices, we refer the reader to Table \ref{oa:measurement} in the Online Appendix, which summarizes the pros and cons of different datasets and measures used in the literature.

%Conceptual challenges
Beyond the issues outlined above, conceptual challenges remain in measuring capex and other investment outcomes. First, should researchers focus on total capex or only incremental capex, as suggested by \citet{richardson2006over}? Second, they must carefully assess misalignments between reported investment and real activity. As noted earlier, disentangling the two is empirically challenging, as reporting and measurement choices can, in turn, influence real decisions \citep{kanodia2007real, zang2012evidence}. Addressing these challenges is crucial for the following reason.
While tax policies generally reward all spending, many are designed to encourage \textit{incremental real} investment that drives growth. This distinction is especially important in cross-jurisdictional studies, where increased spending may simply reflect a reallocation of activity across borders. Such reallocation could bias conclusions, possibly even reversing a policy’s impact from stimulating new spending to merely shifting or altering reported tax bases.

%Suggestions
We offer four suggestions. First, researchers should clearly explain how they calculate investment measures, noting any departures from prior work and explaining how these departures would have changed the interpretation in prior work. Second, pursuing more granular datasets, particularly and preferably administrative data, would be helpful. While access to these datasets can be difficult and restricted, they offer valuable account-level and jurisdiction-specific data.  That said, it is important to note that these data may still be imperfect.\footnote{For example, foreign tax data from IRS Forms 5471 is only available from 2004 and in even-numbered years. Other common datasets like the U.S. Census and Bureau of Labor Statistics have coverage gaps, limiting their completeness.} Despite these challenges, administrative data remains a promising avenue for advancing research.
Third, new measurement approaches should be explored. Public firms often disclose planned investment spending in management guidance, which can provide valuable qualitative context for firm responses. AI tools, such as large language models, could also help extract investment decisions from textual data. Finally, researchers should compare economic magnitudes across studies, potentially using elasticity estimates (see \citealp{zwick2017tax} for a careful benchmarking of results). Without such comparisons, it will be hard for policymakers and researchers to determine whether the observed effects are due to the policy, the data used, or some combination of both. Clear comparisons and discussions relative to prior work will enhance the impact of new research and help integrate findings across the tax literature.


\subsection{Employment and Taxation}\label{section:employment} %[Section by: MARCEL]
\subsubsection{Theoretical Underpinnings and Overview}\label{section:employment_theory}%[Section by: MARCEL]

%Theory Employment versus CAPEX
When relating employment studies to the model of Eq. \ref{equ_theory_coc_maindraft}, one cannot simply treat firms' labor input as the investment variable $I$. 
Since employment expenses are usually fully tax deductible when incurred, such that  $z$  would be 100\%,  corporate income tax rates should not influence firms' input decision (see our detailed discussion in the Appendix and Figure \ref{fig:coc_example_heatmap}).
However, taxes can affect employment through two channels. 

First, employment is linked to investment based on either the complementarity or substitutability of capital and labor. For example, if capital and labor are complements (i.e., if firms need more workers to run newly-acquired fixed assets), then tax incentives lowering the \textit{CoC} will increase both fixed investments and employment. In contrast, if firms substitute labor with new machinery (i.e., automation), employment may decrease after corporate tax cuts \citep*{curtis2022capital}. 

Second, corporate tax policy affects labor to the extent that employees bear some burden of the corporate tax (``tax incidence'').  Firms may pass on their tax costs (savings) to employees, in which case tax increases (decreases) impact workers, largely based on firms' and workers' wage bargaining power. 

Empirical research in finance and economics finds higher corporate income tax rates induce firms to decrease employment, measured with either the number of employees, wages, or both. For example, \cite{giroud2019state}, \cite{ljungqvist2018cut}, and \cite{suarez2016benefits} exploit staggered U.S. state-level corporate tax rate changes to estimate the effects on workers; similarly, \cite*{fuest2018higher} exploit more than 6,000 business tax rate changes across German municipalities to quantify the tax effect on employment.  The use of exogenous variation as well as granular administrative labor data in \cite{giroud2019state} and \cite{fuest2018higher} in particular address  measurement and endogeneity challenges that limited causal analysis of employment effects in prior work. 

For our review, we focus on firm-level employment effects, including the effects of tax changes on the number of workers, as well as compensation paid to rank-and-file workers. Accounting scholars have contributed to this literature in three ways: i) providing additional evidence about the theoretical mechanisms; ii) studying specific types of compensation, and iii) studying additional employment outcomes. A substantial number of accounting papers also examine cross-border employment decisions; we discuss this in tandem with other investment location decisions in Section \ref{section:allocation}.  We exclude papers on executive compensation given the separate accounting literature (and reviews) on corporate governance topics.   

\subsubsection{Evidence on Theoretical Mechanisms}
The economics literature has examined whether capital and labor are complements or substitutes, finding work consistent with both.  For example, \cite*{curtis2022capital} confirm  the view that labor responses are due to the complementarity of labor and capital in firms' investment decisions. In contrast, \cite{Lester2019} finds evidence that domestic employment declined following domestic investment increased among DPAD claimants. More recent work further examines this question in a cross-border context, also producing mixed evidence \citep{garrett2021effects,altshulerjc2023}. The inconclusive results in this literature paint an incomplete picture about under which conditions and after which types of tax policy changes capital and labor behave as complements vs. substitutes.

Further, economists have predominantly examined tax incidence. Incidence for this purpose refers to the extent to which the corporate income tax is economically borne by capital providers, workers, or consumers in the form of lower returns, lower wages, or higher prices \citep{harberger1962incidence}, independent of who is legally liable to remit the tax. Typically, workers bear a greater tax burden if the relative mobility of capital is greater than that of  labor. Workers are estimated to bear approximately 40-60\% of the corporate tax based \citep{arulampalam2012direct, suarez2016benefits, fuest2018higher}.  However, recent evidence indicates that consumers may bear a similar share through higher prices \citep{baker2024consumers}, highlighting the lack of a clear consensus in a large but active literature. 

Recent work in accounting  also examines tax incidence, quantifying the effect on consumers \citep{jacob2023consumers}  and on investment (\cite{jacob2022pit}; see Section \ref{section:firmsreal_investment_nonincometax}).  Other work distinguishes between incidence and implicit taxes \citep{GuentherSansing2023} and studies how tax avoidance and tax incidence intersect \citep{dyreng2022tax}. The premise in \cite{dyreng2022tax} is that firms have an increasing appetite for tax avoidance if the firm bears more of the corporate tax burden. 
While economists have a historical comparative advantage to estimating and documenting questions of tax incidence, the analyses in \cite{dyreng2022tax} show that tax incidence affects firms' tax avoidance -- a topic much studied in accounting. 

We thus encourage accounting scholars to build on these findings in two ways. First, scholars must clearly outline which theories motivate predicted employment effects in their setting given that the traditional theory of investment does not motivate a direct effect on employment. Second, future work should go beyond testing whether employment effects occur after tax rates changes to also include tests of the mechanisms that explain \textit{why} such effects occur.  This evidence in particular is needed to clarify the relatively mixed evidence about the complementarity or substitutability of labor and capital.  

\subsubsection{Tax Effects on Worker Compensation}
Recent work in economics uses matched U.S. firm-worker administrative data to shed light on how tax rate changes impact the distribution of wages paid to workers. After both the DPAD and the TJCA, the evidence points to increased wages disproportionately accruing to top earners, with little to no effect among other ``rank-and-file'' workers \citep*{dobridgedpad2023, kennedyetal2024}. Consistent with this evidence, accounting scholars have demonstrated that firms' tax savings after the TCJA accrue to CEOs due to CEOs' rent extraction opportunities \citep*{andreani2024ceos}.

Only a small number of studies examine the role of other types of compensation, including stock options, pension contributions, and bonuses \citep{brown2008stock, Gaertner2020, Hutchens2022}.  \citet{Hutchens2022} shows that post-TCJA bonuses reduced employee satisfaction, consistent with one-time wage gains being perceived as too small relative to firms’ tax savings.  Beyond this, surprisingly little research exists on the effects of taxation on the \textit{mix} of compensation paid to rank-and-file workers. Many firms offer other types of compensation beyond salaries, including equity-based pay, pension or retirement funds, and bonuses. Understanding how tax impacts the mix of compensation is important because the effective value of tax deductions varies across compensation types and because convexity of the tax code generates different benefits for firms vs. workers.\footnote{For example, because workers tend to exercise stock options when the firm has strong financial performance, stock option deductions are often reported when marginal corporate tax rates tend to be higher, whereas fixed wages provide uniform tax deductions regardless of profitability \citep{babenko2009}.} More research on the benefits and costs of different types and compensation is necessary for  understanding the effect of taxes on ``all parties,'' with a particular lens to better understanding how firms structure compensation packages.  Furthermore, a more complete understanding of the type of worker compensation and the tax treatment thereof would permit examination of tax policies across the wage distribution. 

\subsubsection{Outcomes Beyond Employment and Compensation}
Recent work examines how disclosure of firms’ tax positions shapes stakeholder perceptions, including those of employees. For example, employees react negatively to news about their firm's tax avoidance activities, suggesting that reputational costs among workers accompany tax avoidance activities \citep{lee2021}.  Conversely, ESG funds appear to respond positively to news that a firm claims specific employment-related credits, such as the Work Opportunity Tax Credit paid to firms hiring employees from disadvantaged communities \citep{HutchensWotc2023}. An important issue for future studies of stakeholder perception is to separate the selection of firms that claim the credit from other correlated factors that influence how the labor force sorts into particular firms.

Finally, two papers move beyond firm-level micro measures of employment to examine the aggregate effects of tax changes on country-wide employment growth and wage \citep{Shevlin2019, standridge2023}.\footnote{For studies in economics on the impact of tax changes on aggregate economic activity, including employment, see \cite{romer2010macroeconomic,mertens2013dynamic}, and \cite{cloyne2023short}.} The key insight from this literature relates to measurement.  First, macro measures of tax burdens exhibit a similar negative relation with aggregate investment and employment as in the micro literature, but only to the extent that the tax burdens are measured with effective (not statutory) rates.  Second, aggregate effects of tax policy changes can be estimated using aggregated market response measures.  Using the aggregate capital market response to capture treatment intensity of a tax change enables researchers to estimate broader effects of tax changes and, if measured around short announcement windows, isolate the effect of taxes from other confounding factors \citep{standridge2023,gomezolbert2023}.

\subsubsection{Synthesis and Suggestions for Future Research} 
%Challenges in Reseaerch 
Despite the first-order economic importance of employment, empirical research on the link between business taxes and firms' labor capital decisions is limited, likely due to three main reasons. First, tax policy changes are highly endogenous to employment outcomes, posing challenges in establishing causal relationships. Second, the scarcity of high-quality panel data on employment complicates empirical analysis. Third, firms might not only change employment levels in response to corporate income tax rates (as  reviewed in \citealp{jacob2022review}), but they may also alter the allocation of employment across jurisdictions or the level of wages. 

%How Accounting has Contributed and Can further Do so
An underdeveloped but promising area for accounting researchers is to examine how firms respond with real vs. reporting transactions related to employment outcomes. For instance, if firms shift the classification of employees between full-time, part-time, or contractors, it could affect the evaluation of policy effectiveness in achieving job targets.  Similarly, an open question is whether firms move employees across border to substantiate their income shifting strategies, or if firms instead attempt to reclassify compensation instead as a less costly approach.  Obtaining granular access to tax return data or social security files,  
 if granted by authorities, could enable construction of granular employment measures to address these questions such as in \cite{dobridgedpad2023} and \cite{kennedyetal2024}. Alternative datasets such as Glassdoor and Revelio have also recently provided researchers with new sources for measuring employment activities.
 
Furthermore, open questions remain about the distributional impact of taxation on workers.  Prior corporate governance research provides some evidence about the role of taxation in executives' compensation packages, but to date we have less evidence about distributional effects among rank-and-file workers.  New human capital disclosures in firms' financial statements, as well as data from the Equal Employment Opportunity Commission (EEOC), may provide opportunities to study employment outcomes beyond average wages and employment levels, including differences by gender, hierarchies, or other employee characteristics.  Understanding the heterogeneous impacts of policies along these dimensions would provide a more complete picture of the potential costs and benefits of tax changes along these dimensions.

\subsection{Geographical Allocation of Physical Capital and Employment} \label{section:allocation}%[Section by: MARCEL]
\subsubsection{Theoretical Underpinnings and Overview}

%Theory
A key managerial decision is \textit{where} to invest, and an open question in the literature is the extent to which taxation plays a role in that decision.  Despite many studies on multinational firms, the international tax literature is still very active.  One issue of these studies is that, because national and sub-national governments engage in cross-border tax competition that affects where companies locate, it can be challenging for researchers to fully capture the real response.  Beyond conceptual and policy issues, commonly used data sources limit researchers' ability to precisely identify in which jurisdictions investment occurs. Thus, studying the location of real outcomes across counties is not straightforward due to conceptual, policy, and data issues. 

The canonical model in \cite{hall1967tax} focuses on marginal capex decisions.  \cite{devereux2003evaluating} adapt this theory for the jurisdictional allocation of investment, showing that firms respond to differences in average \textit{effective} tax rates across jurisdictions, which reflects tax incentives, actual economic activity, and profitability of investment (see also \citealp{Mutti2019} and \citealp{chodorow2023tax} for model extensions that incorporate firms' international investment decisions). This prediction extends to the U.S. state and local setting, where firm location decisions are also impacted by firm-specific tax incentives provided by governments \citep{slattery2020evaluating}. 

%Overview and challenge 
The economics literature documents the negative relation between the location of firms' direct investment and host countries' or states' tax rates (e.g., \citealp{devereux1998taxes,feld2003impact,djankov2010effect,feld2011fdi,barrios2012international,Becker2012,giroud2019state}). The accounting field has produced further evidence on firm responses to worldwide and territorial tax regimes, as well as responses at the state and local level. 

\subsubsection{Worldwide and Territorial Tax Regimes}
Several studies in accounting focus on the U.S.'s worldwide tax system (with deferral), which imposed a repatriation tax when U.S. subsidiaries paid dividends to the U.S. parent prior to 2017.  This system, in tandem with U.S. financial accounting rules, motivated firms to retain cash offshore  \citep{foley2007firms,blouin2012us, Graham2010, Ayers2015, DeSimone2019, blouin2012us}.  Several studies use M\&A activity to examine the investment implications of the U.S. worldwide tax system and the corresponding repatriation tax in place prior to 2017. The benefit of using M\&A as a measure of investment is that the location of the transaction (domestic vs. foreign) is more easily observed than other types of investment that may require proprietary or confidential data.  This research finds that the worldwide tax policy increased firms' foreign investment \citep{edwards2016trapped, Hanlon2015repatriation}. Further, the policy created domestic underinvestment problems \citep*{harford2017foreign}. For example, \cite{Harris2018} observe lower levels of domestic M\&A among U.S. MNCs with more complicated foreign tax structures,  and \cite{Bird2017} find an increased likelihood of U.S. domestic targets being acquired by foreign firms due to the U.S. international tax system in place at this time. These studies document investment inefficiencies of the U.S. worldwide tax system prior to 2017: U.S. tax rules motivated foreign, not domestic, investment.   

Additional research documents the economic effects of countries switching from a worldwide to territorial system. Theory suggests that a switch from a worldwide to a territorial system should effectively lower a firm's overall tax burden if foreign tax rates are lower than domestic tax rates.  In this case, firms should increase foreign investment. However, the predictions are not as straightforward given that the switch is often accompanied by other concurrent tax policy reforms, such as transition taxes on historical earnings not previously subject to the repatriation tax, which may decrease investment at least in the short run. 

The literature finds less investment abroad, and increased investment efficiency, after a switch from a worldwide to territorial regime in Japan, the U.K., and (more recently) the U.S. after the TCJA \citep{Arena2015, Amberger2021, Albertus2022, amberger2023initial}. The evidence is consistent with decreased agency-driven investment once cash is no longer ``trapped'' in foreign jurisdictions. However, \cite{liu2020wheredoes} finds \textit{increased} foreign investment after the U.K. tax regime change; the increase in investment occurs in relatively low-taxed countries and thus is consistent with the basic theoretical argument that investment should increase if firms' tax burdens decline. Future research is needed to reconcile these conflicting results, which could in part be due to measurement choices: for example, \cite{liu2020wheredoes} includes both tangible investment, as studied by \cite{Arena2015} and \cite{Amberger2021}, as well as intangible investment, the latter of which may be relatively easier to shift after a tax change. However, more important than replicating or resolving sample and research design differences, the literature needs to understand if differing effects are due to the nature of the tax policy change (was there something specific about one of these countries that is responsible for different effects) or the nature of the impacted firms (more or less prevalence of agency issues)?

\subsubsection{Cross-border Investment and Employment} 
The accounting literature on cross-border location decisions includes two groups: first, studies that examine cross-border investment and employment decisions, and second, research examining innovation box regimes as briefly discussed in Section \ref{section:firmsreal_innovation}.  Focusing on the first, prior work documents that firms' foreign investment is negatively related to host country taxation. \cite{Williams2018} extends the analysis to firm employment, showing that lower foreign tax rates are associated with an increase in both the likelihood and number of U.S. firms' offshored jobs.  The study's sample, which is based on a novel but small dataset of true offshoring events, documents sizeable employment effects of international tax competition in a period when the U.S. had one of the highest statutory tax rates among industrial nations. The tax competition effects on employment are particularly pronounced among those countries offering tax holidays to foreign investors \citep*{fox2021foreign} and are also observed in response to non-income tax incentives, such as consumption taxes \citep{desimone2024multinational}.  

A burgeoning  literature studies foreign investment and employment responses to domestic policy changes, as well as domestic effects of foreign tax rate changes. \cite{Lester2019}, \cite{glaeser2023tax}, \cite{devito2023corporate}, and \cite{hoopes2023africa} use different settings to show that foreign investment and employment increase in response to home country taxation. The key question is why these effects occur, as they are not otherwise predicted by the theory in \cite{hall1967tax}; instead, one must consider other theories that predict firms should invest in the highest NPV project, regardless of jurisdiction (e.g., \cite{hayashi1982tobin}). The literature suggests three possible explanations for the observed effects: \textit{financial constraints}, meaning that firms would not be able to finance the international expansion absent the cash tax savings; \textit{complementarity} of domestic and foreign investment \citep*{desai2005foreign,desai2009domestic}; or \textit{firm scale}, meaning that lower taxes in one country reduce production costs, resulting in overall firm growth  both domestically and abroad \citep{Ohrn2019Depreciation}. \cite{hoopes2023africa} finds evidence consistent with this third mechanism. Aside from this work, the theory and empirical evidence on what motivates foreign investment in response to home country taxes is scarce.

A separate literature examines innovation boxes (or patent or IP box) and the extent to which these incentives in particular motivate cross-border effects. As of 2024, more than 20 countries have innovation or intellectual property (``IP'') boxes. These regimes reduce taxes on intangible-related income, such as license fees, royalty income, and sales revenue derived from IP (for institutional details, see \citealp{Merrill2016,evers2015intellectual,alstadsaeter2018patent}).
The goal is three-fold: to retain and attract innovative activity, to encourage co-location of tangible investment, and to maintain and grow the tax base (reported income) that might otherwise be shifted to lower-tax jurisdictions. 


Several studies examine the extent to which these outcomes motivate new investment spending versus a relocation of existing IP and assets, finding that targeted tax incentives are associated with patenting activity  \citep{bradley2015cross,alstadsaeter2018patent,Schwab2021,shehaj2024corporate}. However, some evidence points to a relocation of patents from other jurisdictions as opposed to new, incremental spending \citep{gaessler2021,schwabandtod}. This implies relatively limited benefit of these regimes in stimulating innovative activity within a country.

Beyond changes in innovation outcomes, the literature also examines the extent of co-location activities, finding  that firms also invest in fixed capital alongside the innovation activities.  Furthermore, IP boxes are associated with increased M\&A activity and more highly compensated employees (\cite{chen2023effect}; \cite{Bradley2021}; and \cite{bornemann2023effect}). However, these co-location effects predominantly occur only within the subset of countries imposing ``nexus'' restrictions that require firms to have real economic presence in the country, or in countries offering relatively larger tax benefits \citep{Bradley2021,chen2023effect}. Because these studies primarily focused on initial years of these regimes prior to OECD rules that now require all IP box regimes to have nexus requirements, the work provides evidence about the effectiveness of such provisions in reducing potential economic distortions \citep{haufler2023attracting}. 

Remaining questions include better understanding the relative benefits and costs of co-location and from which jurisdictions both the intangible and tangible assets are moved, both within the context of IP boxes but also more generally \citep{DeWaegenaere2012}.  More broadly, given that all IP box regimes must now adhere to OECD and EU rules, it is an open question if IP box regimes continue to be a tool for tax competition, or if the effectiveness is more limited given the prevalence  and nexus requirements of these regimes. Open questions also relate to the U.S. Foreign Derived Intangible Income (``FDII'') provision, which is similar in spirit to IP boxes in the sense of providing lower tax rates on certain intangible-related income.  It is designed to encourage U.S. firms to retain IP in the U.S. and source (report) more income in the U.S. than offshore.  Whether these effects have occurred remain open questions.

\subsubsection{State and Local Research} 
The accounting literature contains relatively less evidence on the investment and employment effects of subnational tax competition, such as that across states and localities. As with the international literature, the general finding is that firms locate in the lowest taxed jurisdictions (i.e., \citealp{giroud2019state}).  Furthermore, changes to tax base rules impact both cross-state income shifting and real activity \citep{Li2022, welsch2023effect}. However, subnational competition differs from international competition in two important ways: first, the amount of statutory income tax benefits that state and local governments can provide is naturally smaller in scale, and second, as a consequence, governments increasingly rely on targeted, firm-specific non-income tax incentives to influence business location choices. 

The accounting literature has predominantly focused on studying these non-income tax incentives, including (in the U.S.) abatements for property and sales tax, income tax credits, cash grants, and in-kind infrastructure spending for purposes of attracting or retaining firms in a jurisdiction.  Most of the work examining these incentives occurs in U.S. settings given that these types of targeted incentives are largely prohibited as ``illegal state aid'' in the E.U. The literature finds politically connected recipients are four times more likely to receive state and local awards \citep{aobdia2021} and that they also confer benefits to elected officials providing these incentives \citep{slattery2024governor}. Approximately 60\% of public firms appear to meet the intended job targets \citep{Dongetal2023}, with lower rates among politically connected companies \citep{aobdia2021, Dongetal2023}. Recipient firms appear to engage in more misconduct in the jurisdiction after receipt of an incentive \citep{Raghunandan2024} and use voluntary disclosures to lower public scrutiny \citep{huang2022}. 

In response to the increasing amount of these state and local incentives and to the commensurate public interest in these awards, the U.S. and many states have implemented disclosure regimes to provide both internal government officials and external stakeholders (such as taxpayers and the public at large) greater transparency. Evidence shows that increased disclosure is associated with better job creation outcomes  \citep{desimone2022taxsubsidy} and also with reduced borrowing costs of both the firms and municipalities \citep{Li2024muni}. However, while these papers suggest an important role for disclosure and monitring, the literature has two key challenges.  The first is that firms receiving incentives often differ from non-recipients in several ways, and thus the literature must address selection concerns.  However, this is extremely challenging across all papers on this topic because it is difficult to observe or construct the perfect counterfactual firm absent an incentive.  Second, the literature suffers from lack of complete incentives data, with papers either focusing on the largest ``megadeals'' that often attract press coverage \citep{slattery2020evaluating} or limiting the sample to states where there appears to be sufficient disclosure about the thousands of smaller, but more frequent, incentives \citep{desimone2022taxsubsidy}. Understanding the role and importance of these incentives necessitates a more complete picture of the incentives and the agreements between firms and governments.



\subsubsection{Synthesis and Suggestions for Future Research} 
A key remaining issue for the literature studying real location decisions is understanding the differing mobility of capital and labor and how that impacts firms' responsiveness to cross-border tax incentives.  Previous work provides evidence of strong reallocation of capital and labor across borders \citep{giroud2019state,desimone2022realeffects}, with some evidence that labor may be more mobile than capital. These findings stand somewhat in contrast to the more traditional view that capital is more mobile than labor (e.g., \citealp{keenkonrad2013handbook,pikettysaez2013handbook}).
Consistent with this view, recent research in economics shows that people are not very responsive to tax incentives if they need to move away from home to access lower tax rates \citep{akcigit2022optimal}. 

Collectively, this work suggests that firms' responses to tax incentives regarding labor location are influenced by the characteristics of the workforce and likely vary across time, place, and type of targeted worker. Thus, opportunities for future research exist to better understand these potential effects. In such cases where the workforce is not mobile, tax incentives may induce firms to hire more in local labor markets in one location compared to the other, or possibly even to relocate elsewhere. Given the economic relevance of labor outcomes and the increasing public interest in firms' human capital, future research in accounting can make a contribution by using recent human capital disclosures in firms financial statements to refine measurement and study the varying impact on firms.

\subsection{Intersection of Real Activity and Tax Planning}\label{section:taxavoidancerealeffects}

\subsubsection{Theoretical Underpinnings and Overview}
The studies discussed this far examine firms' real  responses to  particular tax policy change. At the same time, since the call for research on tax avoidance in \cite{hanlon_review_2010}, accounting scholars have provided  abundant evidence that firms engage in tax planning in response to changes in tax policy.\footnote{Following \cite{wilde2018perspectives}, we use the terms tax planning and tax avoidance interchangeably.} However, we have little evidence on whether or to what extent firms' real outcomes are affected by tax avoidance activities \citep{dyrenghanlon2021taxavoidance,jacob2022review,wilde2018perspectives}.  While this relation is difficult to examine given the inherent endogeneity concerns \citep{dyreng2023endogeneity}, evidence is important for not only for assessing the costs and benefits of anti-tax avoidance regulations but also for evaluating the broader equilibrium effects of taxation.  Governments may be willing to tolerate lower effective tax rates for individual firms if the resulting economic growth and subsequently larger tax bases provide net benefits.

%Theory avoidance 
Recent work provides a theoretical framework for the relation between tax avoidance and firms’ economic outcomes. Specifically, \citet*{reineke2023} examine how investment and tax avoidance decisions depend on anti-avoidance rules and enforcement. The model shows that underinvestment occurs due to firms anticipating tax authorities’ audit strategies. However, stricter anti-avoidance rules can mitigate underinvestment problems by increasing investment incentives. Regarding firms’ innovation activities, \citet{reineke2021transfer} show that, in the setting of tax-motivated intellectual property (IP) location decisions, firms sacrifice more efficient pre-tax investment in their home market to benefit from foreign tax incentives. \citet*{dyreng2022tax} propose that tax avoidance levels are linked to tax incidence: firms that bear more of a given tax will both alter real actions and avoid more of that tax if the tax changes. Thus, tax avoidance affects the relationship between tax policy and firms’ real outcomes.

We predominantly review the evidence on (i) the choice of locating in low-tax jurisdictions and (ii) the associated cross-border income shifting as the predominant example of tax avoidance in this literature. This focus reflects that income shifting is the primary ``reporting'' outcome examined in the geographic allocation literature.  

\subsubsection{Tax Haven Investment} 
A key real decision directly related to tax planning is the choice to operate in a tax haven. As the literature across disciplines suggests tax minimization reasons are a primary motivation for multinational firms' tax haven operations \citep{olbertspengel2023}.

The accounting field extensively studies the tax haven choice and has advanced the literature by measuring whether and to what extent firms have a presence in low-tax jurisdictions. Scholars use financial reporting data and firm disclosures, predominantly in Exhibit 21 to firm financial statements, to address this question \citep{dyreng2009using, Law2022}.  While these methodologies permit researchers to identify where firms have a presence, prior work also shows that haven presence is under-reported, possibly in both public and administrative data \citep{DYRENG2016, dyreng2020strategic}. Consequently, researchers must exercise caution and consider potential bias when using these measures to study firm location responses to taxation.  Furthermore, corporate ownership data from Orbis, while extensively covering tax havens if used systematically \citep{olbertspengel2023}, cannot be enriched with financials of subsidiaries due to the lack of reporting mandates in tax havens. Future research could develop alternative  methodologies for identifying firm presence in low-tax countries, possibly using novel data sources such as governmental registries or  websites that reveal placement of employees in particular countries.  

A separate stream of work shows that locating in low-tax jurisdictions is negatively correlated with financial reporting quality and transparency, leading to a higher cost of capital \citep{lewellen2023tax} and likely affecting investment and dividend decisions \citep{Atwood2019}. These findings suggest that haven use has firm value implications beyond tax savings, due to concerns about expropriation and opacity \citep{Bennedsen2018}.  These findings are consistent with broader evidence linking tax avoidance to lower transparency \citep{Balakrishnan2019}.\footnote{\citet{Balakrishnan2019} find that a one-standard-deviation increase in tax avoidance proxies is associated with an approximately 20\% decline in financial reporting transparency measures. While \cite{Balakrishnan2019} do not study subsequent effects on market valuations, \cite{wilson2009examination} and  \cite*{Inger2018} provide  suggestive evidence consistent with the theory that investors discount (value) tax avoidance when they provide less (more) transparent financial disclosures. Collectively, this evidence suggests that avoidance can have  varying real and firm value effects through the effects of a firm's information environment on asset prices (see \citealp*{bond2012real} for a review).}  

Aside from addressing a key issue of selection into tax haven incorporation, broader questions relate to the potential costs that tax haven use impose. These questions include topics related to managerial rent extraction and understanding the extent to which firms compromise governance and transparency when engaging in this type of activity.\footnote{\citet{desai2007theft} provide a theoretical foundation for this relationship. Consistent with this, \citet*{kim2011corporate} show that tax-avoiding firms also engage in obfuscation and rent extraction, leading to the accumulation of bad news and increased stock price crash risk. However, it remains unclear whether these outcomes are solely driven by tax avoidance or by other economic factors correlated with tax avoidance proxies \citep{dyreng2023endogeneity}. Moreover, stock market tests in \citet{Blaylock2016} provide inconsistent evidence, particularly for large U.S. firms with weak governance, challenging the generality of this interpretation.} Research about how investors respond to tax avoidance news is mixed, meaning that there is no clear evidence about whether avoidance and activities in low-taxed jurisdictions in particular, are value-decreasing \citep{Hanlon2009,muller2024investors} or value-increasing \citep{nesbitt2023reexamination,Bennedsen2018}. Future research could clarify the conditions under which haven-related tax avoidance impedes or aligns with value-maximizing real actions.




\subsubsection{Dynamics of Real Activity and Income Shifting}
The literature on income shifting in accounting, finance, and public economics is extensive, starting in the early 1990's and continuing with vigor and much debate today. The evidence generally shows that firms report disproportionately high pre-tax income to lower-tax rate jurisdictions \citep{dyrenghanlon2021taxavoidance}.\footnote{\cite{dyrenghanlon2021taxavoidance} review this literature; see also \cite*{Chen2018} for the association between  income shifting and financial reporting quality, \cite{desimone2016} on how comparability of financial reporting standards and income shifting activities, \cite*{kelley2023just} for firms' reporting responses to anti-income shifting legislation in the TCJA, and \cite{Kohlhase2023} for the use of intra-firm transaction data to show price manipulations as examples of recent and concurrent papers examining income shifting effects on outcomes other than real activities.}  
At the same time, we know that tax incentives such as low tax rates also affect firms' allocation of real resources  (see Sections \ref{section:firmsreal_investment}-\ref{section:allocation}). 

However, the literature has not explicitly addressed the interaction of these effects, nor the dynamics of firms' real and income shifting responses, which we refer to as a ``chicken-egg'' problem. Specifically, the evidence is unclear about whether firms make location decisions (the egg) and layer income shifting on top (the chicken),  whether income shifting incentives (the chicken) affect real resource reallocation (the egg), or whether real activity and income shifting strategies change simultaneously as firms anticipate income shifting opportunities when they decide where to locate.  
We acknowledge that all three scenarios are possible and not mutually exclusive.  We also acknowledge that distinguishing these steps is extremely difficult empirically. We discuss the insights observed from the current literature and also discuss potential opportunities for future research to further distinguish these factors, because a more complete understanding of all three scenarios is necessary to inform the hotly contested discussion about the magnitude of income shifting \citep{blouin2023double,dyreng2023tax, Clausing2020tcja}.

Early work generally infers that firms first make location choices and then layer income shifting on top (``egg then chicken''). This work builds on findings that suggest considerable heterogeneity in the extent to which tax matters for location decisions \citep{Wilson1993}. To the extent that tax is not a primary determinant of location choices, then firms make location decisions based on other business factors (growth opportunities, country- or firm-specific factors), responding to lower tax rates offshore predominantly with reporting activities, such as shifting ``paper'' profits \citep{Slemrod1992}. This perspective is also reflected in the seminal empirical income shifting models based on \cite{hines1994fiscal} and \cite{huizinga2008international}. These models take firms' real factor allocation decisions as given and model income as an outcome of  key production factors, often based on a Cobb-Douglas production function. The models then detect income shifting based on a separate loading on the tax rate variation added to the model.  

However, recent evidence suggests that income shifting benefits impact how much and where firms invest and employ workers (``chicken then egg''). Specifically,  studies suggest that firms shift real activity into low-taxed jurisdictions to substantiate their income shifting activities \citep{Williams2018, drake2022foreign, desimone2022realeffects}. A possible mechanism behind these findings is the increased scrutiny by the tax authorities about the magnitude of income shifting and the increasing importance of demonstrating economic substance in foreign jurisdictions under stricter transfer pricing guidelines.\footnote{For example, in its action plan on Base Erosion and Profit Shifting (BEPS), the OECD stated that  ``a re-alignment of taxation and relevant substance is needed to restore the intended effects and benefits of international standards,'' where presumably ``substance'' refers to capital and labor.  As a specific example, the U.S. Senate scrutinized Caterpillar's income shifting transactions, in part because they reported 85\% of profits in Switzerland, even though only a tiny fraction of employees worked there (65 out of 8,300 parts specialists) \citep{drake2022foreign}} 
These results suggest that the benefits of income shifting (or the costs of tax enforcement) are so great that they must exceed the costs of increasing real activity in low-taxed jurisdictions. An alternative explanation is that foreign jurisdictions offer both operational and tax benefits, such that firms can concurrently optimize across multiple dimensions (``chicken and egg''). This last explanation is consistent with the observed increase in multinational investment and employment in Ireland, Switzerland, and Singapore in recent years.

Furthermore, other recent studies also show that real activity in low-taxed jurisdictions and total investment at the firm level can decrease as stricter tax enforcement mitigates income shifting opportunities \citep{DYRENG2016, Fox2022, chow2023cross}. Consistent with the idea that income shifting affects subsequent investment, additional work shows that income shifting comes at the expense of efficient international investment \citep{desimone2022incomeshiftinginvestment},\footnote{As \cite{desimone2022incomeshiftinginvestment} note, establishing a direct causal link remains challenging, and further research is needed building on theoretical frameworks as, for example, in \cite{reineke2021transfer}. Consistent with \cite{desimone2022incomeshiftinginvestment}, \cite{traini2022aggressive} show that higher tax avoidance leads to poorer labor investment decisions. %Similarly, \cite{Chyz2018} finds that greater tax avoidance increases the likelihood of CEO turnover, possibly due to reputational risk or the negative stock market outcomes linked to tax avoidance, which affect CEO labor outcomes.
} and that reductions in income shifting opportunities have negative effects on investment and employment in the home country \citep{suarez2019unintended}. While this literature has made first steps for studying the intersection of real and reporting responses, extant work has not clearly disentangled the two outcomes of interest.  Thus, more clarity is needed about the dynamics of these actions in the foreign countries, the impact on investment efficiency, and the potential negative consequences for home country investment.   


\subsubsection{Income Shifting Mechanisms}\label{section:incomeshiftingmechanisms}
To understand the dynamics of income shifting and firms' related real responses, one approach is to study which corporate transactions facilitate income shifting. The literature suggests that income shifting occurs predominantly through three channels. 
The first is intercompany financing and dividend payments \citep{Dyreng2015, murphy2023foreign}, which is not directly related to our focus on real outcomes.
Second, the strategic placement of (intangible) assets  facilitates the sourcing of income to a low-taxed jurisdiction. The literature that best speaks to this examines IP boxes, where papers look at changes in the reported level of income after an IP box regime is in place \citep*{schwab2021thinking,koethenbuerger2019intended,bornemann2023effect}. However, this question is largely unanswered because the existing results could possibly be explained by the concurrent investment and employment changes that impact a firm's production function, rather than income shifting (which is captured by the sensitivity of income to taxation, not necessarily from a change in the level of income). Future work employing empirical strategies that can separately measure income shifting and innovation activity responses is needed to further advance this work.

Third, several papers provide new insights related to the strategic pricing and timing of intercompany goods and services trade for tax-motivated income shifting \citep{hebous2021your, chow2023cross,olbert2023private, dobridge2024ipos}, all of which use novel or administrative data. For example, \cite{chow2023cross} uses shipping container details to show that bilateral information exchange agreements intended to reduce  income shifting are associated with large reductions in U.S. firms' imports, meaning that reduced income shifting opportunities also reduce intercompany trade of goods.  Using IRS administrative data, \cite{dobridge2024ipos} show that firms' propensity to own a haven subsidiary and have a cost-sharing agreement for income shifting purposes substantially increase around the time that firms go public and that income-shifting-related payments shortly follow the IPO. Consistent with the evidence in the private equity setting \citep{olbert2023private}, these findings suggest that firm expansion due to greater capital access is also associated with greater cross-border income shifting. To advance this line of work, future research should focus on improving the measurement of intrafirm versus external party transactions to isolate real changes related to income-shifting incentives. Additionally, researchers should aim to more precisely determine the timeline of corporate actions in relation to implementing real changes and reporting tax bases in different jurisdictions. 

%%%%%%%%%%%%%%%%%%%%%%%%%%%%%%%%%

\subsubsection{Synthesis and Suggestions for Future Research}
While the studies reviewed in this section have clearly increased the awareness of the co-movement of tax planning and real responses, we lack evidence to better understand the dynamics of these effects. For tax avoidance in general, and particularly in the setting of cross-border income shifting, addressing this ``chicken-egg'' problem is important for three reasons.  

The first is policy-related: the interpretation of the interrelationship between real activity and income shifting, which can depend on the setting and firm type, matters for the policy implications.\footnote{For instance, a government may intend to offer income shifting opportunities if firms invest in their jurisdiction, while the same government may want to impose anti-avoidance legislation if a firms' global footprint is fixed but income shifting activities are increasing.} Specifically, many recent policies attempt to reduce tax-motivated income shifting on the basis that the shifting is only ``on paper'' and thus lacks real substance. To the extent that income shifting follows substance, it may directly alter researchers' inferences and also change the type of policy intervention. Understanding this relationship is closely tied to the ongoing debate about the magnitude of income shifting. \citet{blouin2023double} demonstrate how accounting researchers can significantly advance the field by addressing measurement and geographic attribution issues in estimating income shifting.\footnote{The key message from \citet{blouin2023double} is that researchers must carefully interpret reported pre-tax income to avoid double-counting by upper-tier owners and ensure proper attribution to the correct jurisdiction. Their findings suggest that annual income shifting is roughly \$7-\$11 billion, about one-tenth of prior estimates. These issues also affect macroeconomic data, as administrative data often rely on firm-level reports and are similarly impacted by double-counting and misattribution.} Similarly, accounting researchers are well-positioned to combine and analyze various data sources to distinguish income attached to real activity from ``paper'' shifting, which is the target of anti-avoidance rules.

The second relates to externalities: if firms re-allocate capital and labor to substantiate income shifting strategies, it may come at the cost of the most efficient pre-tax allocation of resources to international markets. However, without addressing this simultaneity problem, it is challenging for researchers to evaluate the efficiency of firms' location decisions. 

The third is a research design issue: The field needs careful measurement of reported income and related real activities generating this income \citep{blouin2023double,dyreng2023tax}. Further, it is unclear how to interpret economic results by studying income shifting as the outcome and controlling for capital and labor input factors if these real factors are simultaneously affected, as this ``bad controls'' problem may bias coefficient estimates. Addressing this issue is challenging. While there is no obvious methodological fix, we do not intend to discourage research in this area.  On the contrary, we look forward to research that makes a contribution by actively addressing the issue and proposing methodological advancements and relying on theoretical frameworks. Recent international tax policy changes, such as those included in the TCJA and future changes in response to Pillar 2, may afford opportunities for this analysis. A promising starting point is to study real and tax avoidance outcomes separately within the same setting that has clear predictions (e.g., \cite{gschossmann2024location}). Further, we encourage researchers to acknowledge the simultaneity issue and, at a minimum, conduct diagnostic tests to assess the issue in light of their inferences.  

%Tax avoidance and information environment

Finally, it is unclear to what extent these real and general tax avoidance or income shifting decisions impact a firm's information environment. Several studies show significant links between firms’ tax avoidance and financial reporting quality, highlighting how tax avoidance affects the information environment and investor perceptions. However, the evidence is inconclusive as \cite*{Chen2018} find a negative relation between cross-border income shifting and corporate transparency and \cite{Huang2020} finds a positive association in a similar setting.   
Furthermore, an open question remains whether firms adjust investment strategies based on market feedback related to tax avoidance reporting behavior (e.g., \citealp{bond2012real,jayaraman2020should}).
Finally, there is no direct evidence on which specific corporate actions drive changes in investors’ perceptions about firm risk and firm value related to tax avoidance.  Research on these issues would also benefit from a clear link to the broader accounting literature about opacity and transparency.


%TA in general
In sum, there is limited direct evidence on how tax avoidance affects investment and other real outcomes, though some studies suggest high-quality governance helps balance tax and investment strategies \citep{Armstrong2015}. Given the endogenous relationship between tax avoidance and firm outcomes \citep{dyreng2023endogeneity}, future research should focus on settings where changes in tax avoidance opportunities are independent of future investment decisions. This issue is especially relevant when studying effective tax rates, market power, and the decline in multinationals’ foreign tax rates, which may be confounded by simultaneous performance or size factors \citep[see, e.g.,][]{rego2003tax,Dyreng2017}.
To address the issue, building on theoretical frameworks, such as \citet{reineke2023}, will help establish more robust inferences.
Further, future research could use isolated policy changes that impact tax savings to provide direct evidence on how these savings influence firms’ real decisions and competitive positions, as indicated by indirect evidence in Section \ref{section:externalities}. This would have implications for tax policy and antitrust regulations, particularly regarding the role of taxes in shaping competition, a key driver of growth and innovation.  


\section{Type of Investments and Responses to Disclosure Regulation} \label{section:investment_types}%%[Section by: BECKY]

\subsection{Risk-taking and Statutory Tax Rate and Base Rules} \label{section:risktaking}%%[Section by: BECKY]
\subsubsection{Theoretical Underpinnings and Overview}
The theory of how taxes impact the riskiness of investment is outside of the \cite{hall1967tax} framework.  \cite{Domar1944} provide the theoretical foundation for  examining the investment choices of individual taxpayers. Specifically, they study how the tax system, with both a tax rate \textit{t} and a tax loss offset rule, affects investment \textit{I}. 
More generous loss offsets effectively lead to  risk-sharing between the taxpayer and the government as follows: the government enjoys a payoff if the project is profitable  by taxing the return at \textit{t}, but it also shares the downside risk if the project generates a loss because it grants the taxpayer a refund to offset prior or future taxable income. Because this risk-sharing reduces the variance of the project below the preferred amount, the taxpayer selects \textit{I} with a higher level of \textit{ex ante} risk  to restore the expected return to the pre-tax level.\footnote{It is important to note that this literature  is distinct from studies examining whether tax avoidance directly affects the risk of the firm. The results of this literature are mixed: some studies find a negative or no association (e.g., \citealp{goh2016effect,Guenther2017}), suggesting that cash flows from tax avoidance are less risky than overall cash flows. Others find a positive association (e.g., \citealp{Heitzman2019,Lewellen2021,Donelson2022}), indicating greater tax uncertainty and cash flow risk for tax-avoiding firms \citep{Dyreng2019}. \citet{hutchens2024tax} suggest that the differing findings are partly due to notable heterogeneity in the cross-section of firms.} \cite{Langenmayr2018} adapt  this theory to the business tax setting; their model's predictions show that more generous tax loss carryback rules, and, to a lesser extent, more lenient carryforward rules, have a direct effect of increasing corporate risk taking. Higher tax rates discourage risk taking if tax loss carryback and carryforward rules are relatively strict.  Below we discuss the empirical literature testing this theory for i) corporate taxes and ii) individual taxes affecting executives.   

\subsubsection{Corporate Taxes and Corporate Risk-Taking}
\citet{Langenmayr2018} provide empirical evidence from the European setting confirming the two theoretical predictions from their model. Specifically, they show that corporate risk-taking is increasing in the tax loss period, with the largest effects for loss carrybacks. They also document that the effect of the tax rate hinges on the extent to which the company expects to claim a tax loss offset.  The literature confirms this relationship using variation in tax rates and tax reporting regimes across the U.S. \citep{Ljungqvist2017, welsch2023}, among community banks \citep{glenn2021effect}, among thoroughbred racing businesses \citep{ferguson2023}, and examining the impact of loss carry-forward rules on firms' bankruptcy choices \citep{olbert2023} and the timeliness of investment decisions \citep{hillmann2024effect}. 
%Synthesis

Despite this evidence, the mechanisms and broader consequences of these findings deserve more attention.  
Some work focuses on the effect of tax rates without explicitly considering the role of losses (e.g., \citealp{Ljungqvist2017}), even though theory suggests that losses are the key channel behind risk-taking effects. Thus, it is difficult to interpret these findings, and it is unclear whether risk-taking is driven by firms with current or expected losses, or if   profitable firms also respond as they value the ``insurance'' provided by the tax system. Understanding this heterogeneity is important as loss rules and risk-taking behavior have competitive externalities that influence market entry and exits, a key determinant of aggregate growth and innovation \citep{gentry2000,da2011entrepreneurship, Bethmann2018, olbert2023}.


\subsubsection{Executives' Individual Taxes and Corporate Risk-Taking}
A related literature examines risk-taking  based on managers' individual tax considerations.  \cite*{Armstrong2019} show that managers' tax rates are positively related to corporate risk-taking; \cite{Yost2018} focuses on the amount of CEOs' unrealized capital gains attributable to owning firm shares and finds that this tax burden is negatively associated with corporate risk-taking.  The intuition in \cite{Yost2018} is that CEOs' unrealized capital gains tax liabilities overexpose the executives to firm risk, thereby reducing their incentive to invest in risky projects.  The differing effects across these two types of individual taxes raises questions about the relative importance, or ``pecking order'' of tax policies: When and how do managers' capital gains and ordinary income tax burdens  drive corporate risk-taking?  Furthermore, are these effects more important than corporate tax policies?  Evidence that further considers ``all parties'' (both the firm and shareholders in this case) would resolve these questions.

\subsubsection{Synthesis and Suggestions for Future Research} 
The literature shows that tax systems influence business risk-taking, but several important questions remain open. First, while \citet{glenn2021effect} and \citet{ferguson2023} provide new insights about risk-taking among smaller businesses, it is unclear if these effects generalize to a broader group of flow-through businesses in which tax loss offsets are determined at the owner level.  Specifically, it is unclear whether or to what extent these firms consider tax losses when making their investment decisions. Second, the literature needs a better understanding of the magnitude of effects across taxing jurisdictions to evaluate the relative importance of differing tax policies on risk-taking outcomes. Finally, as prior work has predominantly examined consolidated firm-level investment proxies, there is little evidence on specific corporate decisions involving uncertainty, such as entering new markets or exiting old ones (e.g., \citealp{Bethmann2018,buhrle2023value, olbert2023}). Future research is needed to understand the role of tax loss policy on investment in emerging assets with risky outcomes, such as AI or the development of green technologies. 

\subsection{Indirect Real Effects: Externalities of Tax Policy through Firms' Responses}\label{section:externalities}%Ownership of section: MARCEL so far entirely
\subsubsection{Theoretical Underpinnings and Overview}
%Theory indirect effects
Tax policy can have (often unexpected) indirect effects if firms affected by tax changes alter their real actions in a way that impacts their competitive position. Theoretical predictions for how these changes indirectly impact firms depend on the nature of competition and product markets \citep{tirole1988theory}. For example, tax cuts may induce affected firms to increase investment or marketing, improving their competitive position. In a perfectly competitive market, firms that are indirectly affected might respond by cutting costs or investment, or alternatively increasing investment to remain competitive. Tax benefits could also enhance the market power of directly affected firms, potentially reducing the financial performance or even forcing competitors out of the market. This section discusses the emerging accounting literature that explores how tax policies and firms’ responses to these policies impact competition.

\subsubsection{Competitive Effects of Taxation}
\citet*{gaertner2020making} provide indirect evidence of tax reforms’ competitive externalities by documenting negative stock price reactions of non-U.S. firms around the 2017 U.S. TCJA. Their paper suggests that investors expect U.S. tax cuts to harm the future profitability of foreign competitors, especially Chinese firms in steel, business equipment, and chemical manufacturing. Recent studies provide more direct evidence on such competitive externalities of tax cuts in the U.S. \citep*{Kim2021,Donohoe2022} and European settings \citep*{glaeser2023tax}.
%\citet*{Kim2021} find that U.S. manufacturing firms exposed to foreign tax cuts face increased competition, leading to lower price-to-cost margins but higher capital expenditures. \citet*{Donohoe2022} show that the 2004 U.S. repatriation tax holiday decreased the operating performance of firms competing against peers benefiting from low-tax cash repatriations. \citet*{glaeser2023tax} document that European tax rate cuts increase exports and subsidiary-level investment by domestic firms, while reducing employment at firms exposed to foreign tax cuts through competition.

The key questions for this line of research are which specific taxes drive cross-border and domestic competition, and to what extent, particularly when benchmarked against other types of regulation. \citet{Kim2021} provide a good starting point by showing that U.S. firms in industries more exposed to foreign tax cuts use more competition-related language in their 10-Ks. \citet{Donohoe2022} offer cross-sectional evidence supporting the interpretation that windfall tax benefits encourage predatory behavior by tax-advantaged firms. \citet{glaeser2023tax} demonstrate that the negative effects on indirectly affected firms are significant enough to persist at the industry and country levels.
However, questions remain about the specific mechanisms driving these baseline effects. For example, it is unclear how firms use tax savings to impact their peers. Furthermore, we lack evidence on the broader economic consequences of these activities, such as the net effect of directly affected firms’ actions and the externalities on peer firms, which are of primary interest to policymakers \citep{breuer2021does}.
 
%%OLD PARAGRAPH ON MARKET SIZE PAPERS:
%Recent studies examine whether large firms with market power pay relatively low taxes, or whether the tax system provides opportunities for tax avoidance that enable predatory behavior, leading to the rise of monopolistic firms. \citet{gallemore2023does} and \citet*{gaertner2023firmsize} use data on all U.S. public firms over an extended period to analyze correlations between effective tax rates and proxies for firm size and market power, but they reach different conclusions. \citet{gallemore2023does} find that U.S. “superstar” firms, identified by market share and profitability, do not engage in more tax avoidance than their peers. In contrast, \citet*{gaertner2023firmsize} show that large firms generally benefit from lower effective tax rates, which contradicts the political cost theory suggesting larger firms face higher tax rates \citep{zimmerman1983taxes}.
%\citet*{martin2022corporate} examine this question more systematically by proposing a model and an empirical strategy using IRS audit likelihoods as an instrument for tax avoidance. Their results suggest that greater tax avoidance increases firm sales by lowering marginal costs, allowing tax-avoiding firms to gain larger market shares. Yet, it is unclear how the empirical design effectively addresses the correlation between firm size, audit likelihoods, tax avoidance and further economic consequences. 

%For details around these studies, see earlier versino *UNC version)

\subsubsection{Synthesis and Suggestions for Future Research} 
The young work reviewed in this section shows that tax cuts in one jurisdiction influence the investment and performance of firms in other markets, particularly those facing greater competition. Future research could investigate how firms use tax savings for predatory behavior by employing tools like textual analysis of management guidance and voluntary disclosures in short windows around tax policy announcements. This approach would also help quantify the extent to which taxes influence competition compared to other types of regulations (e.g., \citealp*{armstrong2024measuring}). New evidence in this area would also help reconcile different results in recent work that re-examines whether large firms engage in more tax avoidance and whether the tax system favors large firms that can use tax savings to exploit their market power  \citep{zimmerman1983taxes,martin2022corporate,gallemore2023does,gaertner2023firmsize}. Additionally, further evidence is needed to evaluate the large economic effects reported in prior studies and to investigate whether tax-induced competition  increases welfare in the long run, even if it negatively affects domestic employment in the short term.

Further, the literature on indirect real effects of taxation has focused on foreign tax policies. However, disentangling the tax effects in these settings can be challenging as compared to other forces that drive cross-jurisdictional competition. Thus, we suggest considering settings in which peer firms face similar economic and other regulatory environments but differ in their exposure to certain tax policies to the extent that it helps to better isolate the role of taxation on competition. 
Further, accounting researchers are well-suited to expand this field to the tax enforcement as well as disclosure and contracting setting. As existing research shows significant direct real responses to taxation  and tax disclosure mandates or enforcement, these policy changes likely have broader indirect, understudied consequences (e.g., local tax enforcement impacting bank lending or economic activity overall, \citealp{GallemoreJacob2020,gallemore2025corporate}).

Future research by accounting scholars in this area has significant potential for impactful contributions for two reasons. First, assessing the overall consequences of taxation requires understanding both its direct and indirect effects \citep{huber2023estimating,berg2021spillover}.  Tax accounting researchers can leverage expertise from related subfields, such as strategic disclosures and the spillover effects of disclosures, to develop methods for measuring firms’ exposure to specific peer firms, as well as the corresponding incentives and potential responses as these peer firms respond to tax changes. Second, researchers can possibly provide precise effect sizes and address underexplored issues by drawing stronger causal inferences. This is because tax policies are often designed with directly affected firms in mind, studying externalities among indirectly affected firms or broader units of observation is less prone to obvious endogeneity concerns. Further, this research is well-suited to using exposure designs based on shift-share approaches, or so-called Bartik instruments \citep*{goldsmith2020bartik,breuer2022bartik}. As demonstrated in \cite{Kim2021}, \cite{Fox2022}, and \cite{glaeser2023tax}, firms, industries, or regions often exhibit ex-ante heterogeneity in their exposure to specific tax policy changes, which allows such research designs  to control for classical confounding factors correlated with changes in tax policies.\footnote{Using a similar empirical strategy, \cite*{armstrong2019strategic} show that U.S. firms reduce their effective tax rates after peer firms have easier ability to engage in tax avoidance due to favorable foreign tax cuts in Ireland. Although this study does not specifically examine real outcomes or operating performance, lower ETRs ultimately increase net margins,  highlighting possible competitive externalities (see also \cite{Bird2018} and \cite{Bauckloh2021} for consistent evidence using other empirical settings).}



\subsection{Taxes and Sustainability Outcomes}\label{section:sustainability}%%[Section by: MARCEL]
\subsubsection{Theoretical Underpinnings and Overview}
%see also \citealp{bray2021corporate} and \citealp{metcalf2021carbontheorypractice} for reviews
%Theory intro
Firms’ tax affairs are central to the ESG discussion, particularly for those subject to mandatory disclosure regulations and pressure from stakeholders to voluntarily report on ESG performance. However, there is currently no established theory or framework that explains how stakeholders, shareholders, or managers incorporate tax considerations into broader ESG strategies \citep{bonham2022motivatingesg}.\footnote{\citet{bonham2022motivatingesg} propose a theoretical model outlining conditions under which tax policy can drive ESG outcomes. We also acknowledge keynote speaker Alenka Turnsenk, participants at the 2023 LBS-Stanford Global Tax Conference, and Irina Luneva for their valuable input on the framework regarding firms’ ESG and tax strategies. }
Within the framework of our review, we identify three ESG issues that span firms' responses to taxation. 
%This lack of framework is likely due to the recent and dynamic nature of the issue and the wide range of different but interrelated aspects of ESG and business taxation.  


%First taxes as S
First, taxation is commonly used as a policy tool to promote sustainable business practices, particularly for environmental outcomes. A tax on outcomes with negative ESG characteristics reduces the marginal return on investment in those areas (i.e., a lower $MPK$ in Eq. \ref{equ_theory_coc_maindraft}). Such a Pigouvian tax, $\tau_{esg}$, would apply to specific inputs with adverse ESG effects.
Currently, environmental taxes are the primary tool for sustainability-related tax policies. By taxing either firms’ emissions or greenhouse gas-intensive products (e.g., a carbon tax on gasoline), governments increase the cost of emissions-intensive inputs or reduce the net revenue from emissions-intensive outputs. Relating this to Section \ref{section:framework_basic}, a carbon tax reduces marginal returns via higher prices and thus lower profit margins (i.e., a lower $MPK$ in Eq. \ref{equ_theory_coc_maindraft}), while an emissions tax, $\tau_e$, on inputs proportional to investment, $I$, raises the cost of capital by entering Eq. \ref{equ_theory_coc_maindraft} negatively.
To mitigate this cost, firms are incentivized to reduce emissions, either by switching to greener technologies (as in \cite*{krass2013environmental,acemoglu2016transition} and \cite*{shapiro2023macroeconomic}) or by reducing output and investment. Firms may also avoid these taxes through strategic reporting or regulatory arbitrage (i.e., reallocating activities, often referred to as “carbon leakage”).\footnote{Executives view carbon taxes as a powerful driver of change in firms' environmental practices \cite{stroebel2021you}, and many governments have implemented such policies to induce technological change. Although broader ESG-related taxes or subsidies are conceivable, their complexity and political divide make them unlikely in the near future \citep{bonham2022motivatingesg}. Conceptually, such an ESG tax, $\tau_{esg}$, could apply to labor inputs deviating from desired workforce compositions, also increasing the cost of capital in Eq. \ref{equ_theory_coc_maindraft}.}
%\cite{Karpoff2022}  argue that direct policy tools are the more effective tool to induce sustainable corporate behavior with respect to environmental pollution. Disclosing tax burdens  and stakeholder reactions to these disclosures can elicit real effects.

Second, governments, shareholders, and other stakeholders increasingly consider firms' tax payments as a social outcome of ESG activities, since tax payments contribute to public finances that ultimately serve society.\footnote{ For example, the Norwegian sovereign wealth fund, a powerful global institutional investor, has divested shares of companies that are opaque due to their allegedly aggressive tax planning strategies \citep{reuters_tax_transparency_2021}. Legal scholars see tax strategies as an integral part of ESG behavior \citep{hongler2021tax}, and this concept is seemingly shared among industry practitioners (see, e.g., PwC's thought pieces and website marketing material \href{https://www.pwc.com/gx/en/services/tax/publications/tax-is-a-crucial-part-of-esg-reporting.html}{here}. The KPMG tax advisors \cite{janowak2021t} provide some evidence on institutional investors driving responsible tax strategies in investee companies. \cite{cowan2023esg} even question whether ESG-minded firms should make use of tax incentives provided by local governments as exploiting the associated benefits can constrain local public finances.} As firms are increasingly evaluated on their tax contributions as part of ESG performance, they may reconsider their tax strategies, which could affect both the total taxes paid globally and the distribution of tax types and amounts across jurisdictions. As discussed in Sections \ref{section:framework_extended} and  \ref{section:taxavoidancerealeffects}, these changes often correlate with real outcomes such as the level and allocation of investment.

%Mandatory and Voluntary disclosures
Third, and related to the second point, credibly communicating that a firm pays its perceived fair share of taxes is becoming an increasingly important element of corporate strategy. This communication gains importance as emerging ESG-related transparency regulations make firms’ tax affairs more publicly observable. This marks a departure from traditional tax ``secrecy'' with   only limited tax information observed in financial statements at the consolidated level. 
The rationale behind these disclosure mandates is that greater transparency allows market forces -- such as investor and stakeholder pressure -- to drive corporate change \citep*{christensen2021mandatory}. For tax-related disclosures, the assumption is that tax authorities or broader stakeholders will use this information to encourage firms to contribute their “fair share” to public finances. 
Investors and academics have also advocated for greater voluntary tax-specific disclosures \citep{rajgopal_2020,kaplan2021fixesg,rajgopal_2022} and the development of standards like \href{https://www.globalreporting.org/standards/standards-development/topic-standard-for-tax/}{GRI 207}, which some firms are choosing to adopt.
As a result of this trend, firms are likely to reassess their tax disclosure strategies, which may also lead to real effects \citep{roychowdhury2019effects}.



%Overview
We review the early work in these three areas, focusing on the emerging literature on firm responses to green taxes, with a brief overview of the economics literature which was first to study the effects of environmental taxation at the more macroeconomic level. Accounting researchers can build on this work through focusing on financial-related measurement and transactions. Next, we discuss research on tax and disclosure outcomes related to ESG considerations and regulations that may also drive real corporate change. Finally, we outline future research directions, based on the idea that firms should integrate tax and disclosure strategies with operational changes that enhance sustainability and long-term firm value \citep{edmans2023end,edmans2024rational}.



 
\subsubsection{Evidence on the Effects of Environmental Taxes}
%\footnote{As we limit the scope of our review in this section to studying aggregate outcomes as well as direct firm responses, we do not focus on papers studying price effects that can contribute to per-capita changes in emission reductions, such as via gasoline taxes (e.g., \citealp{davis2011estimating}). See also \cite{lin2011effect} for an overview.}
%\hl{Big studies: for motivation, not for actual review }\cite{Greenstone2014,Hanna2015} que question, tax policy one of them \cite{metcalf2019economics}, understanding firm responses crucial
%Tax policy theory: \cite{barrage2020optimal}, \cite{Iverson2021},  

%\hl{maybe for footnote}: NTJ issue with qualitative and policy notes: \cite{Williams2015INCIDENCEcarbon} for incidence, NTJ issue \cite{Tuladhar2015,Jorgenson2015,Rausch2015}
Both the macroeconomic and firm-level evidence can be broadly categorized into (i) the effectiveness of tax policies for reducing emissions, (ii) the consequences  for other economic outcomes such as investment, including green technologies, and (iii)  carbon leakage, a form of tax avoidance in the environmental setting.

%economics studies
A substantial economics literature has explored the elasticity of pollution in response to environmental taxes or carbon prices since the 1970s (\citealp{rafaty2020carbon}, \citet{metcalf2021carbontheorypractice}, and \cite{timilsina2022carbon} for reviews, and \cite{hahn2024welfare} for a welfare analyses of U.S. policies). The consensus is that higher carbon prices reduce pollution \citep{rafaty2020carbon,andersson2019carbon,metcalf2022macroeconomic}, although effect sizes vary, and generalizability is limited due to measurement and identification challenges.\footnote{Some studies report no significant effects, and \citet{Leslie2018} even finds that carbon taxes can \textit{increase} emissions in certain market structures.}
Macroeconomic research also studies the impacts of carbon pricing, showing modest positive effects on GDP and employment growth \citep{metcalf2022macroeconomic}, no long-term impact on inflation \citep{konradt2021carbon}, and labor shifts from carbon-intensive to energy-efficient firms \citep{dussaux2020joint}. \citet{kanzig2022unequal} finds that regulatory changes in the Emission Trading Scheme (ETS) temporarily reduce economic activity, disproportionately affecting low-income households, while also prompting some firms to adopt greener technologies.
%Carbon leakage
The modest effects of environmental taxes on pollution and economic activity, despite the high theoretical cost, raise the question of whether firms engage in regulatory arbitrage. One example of such arbitrage is carbon leakage, where   emissions shift from regulated markets (e.g., countries with carbon taxes) to unregulated ones. \citet{aichele2015kyoto} and \citet{naegele2019does} find suggestive evidence of carbon leakage using sectoral bilateral trade flows following the Kyoto Protocol.



 
%\subsubsection{Firm Responses to Environmental Taxes}
%Firm-level Pollution
At the firm level, evidence suggests that firms reduce emissions in response to higher carbon prices from taxes or cap-and-trade systems \citep*{Martin2014,dussaux2020joint,colmer2024does}. A key question for future research is to refine effect sizes and understand under what conditions firms reduce emissions in response to environmental taxes. For example, \cite{alonso2024cross} show that the EU's CBAM creates predictably larger costs for European firms and that the affected firms are not fully able to pass on the costs to other firms in their supply chain. As firm responses seem more heterogeneous than the responses to income taxes, and attitudes toward climate policies vary widely \citep{stantcheva2024climate}, more evidence on their effectiveness will be crucial for informing policy. 

Two recent studies stand out for their quantification using novel firm-level data and rigorous identification strategies. \citet{martinsson2024effect} find that Swedish firms reduced CO2 emissions by 2\% for every 1\% increase in carbon prices due to a higher Swedish carbon tax over 30 years, implying that 2015 emissions would have been 30\% higher without the tax. In contrast, \citet{erbertseder2023effective} document a smaller effect of a 1.2\% reduction in emissions following the introduction of a local emissions tax in Spain.\footnote{Direct comparisons between these studies are difficult. \citet{erbertseder2023effective} study the introduction of an emissions tax using control firms with no tax burden, while \citet{martinsson2024effect} analyze changes in tax rates over time. The emissions tax in \citet{erbertseder2023effective}, ranging from EUR 9 to EUR 50 per ton, is relatively high compared to OECD averages, but their results remain modest. Another factor is the difference in methods: \citet{erbertseder2023effective} use satellite data to approximate nitrogen oxides (NOx) emissions, while \citet{martinsson2024effect} rely on plant-level administrative filings. 
} These modest firm-level effects align with lower bound estimates from macroeconomic studies \citep{metcalf2022macroeconomic,rafaty2020carbon}, but the wide variation in firm-level responses indicates significant heterogeneity in reaction to environmental taxes.


%\cite*{Martin2014} use plant-level administrative data from the U.K. to show that firms   affected by the 2001 U.K. carbon tax reduced CO2 emissions by 8-22 percent compared. % to plants paying a benefiting from an 80 percent discounted  tax.
%\cite{dussaux2020joint} uses survey and administrative data on French manufacturing firms and finds that higher energy prices lead firms to consume less fuels and switch to more energy-efficient inputs. %While these results are not directly attributable to environmental taxes, additional tests show that carbon taxes likely induce such effects.
%\cite*{colmer2022does} exploit regulatory carbon price changes in the ETS and document that  manufacturing firms in France  reduce carbon dioxide emissions by 14-16 percent but do not seem to decrease output or employment. %\hl{moreo details, also be clearer on empirical strategy and magnitudes }. They use french administrative data


%Investment   %Technological change%Technological change
An important research question in this area relates to the mechanisms driving changes (or lack thereof) in emissions. Firms may respond to carbon taxes in at least three ways to reduce emissions, but direct empirical evidence remains scarce and inconclusive.
First, firms may reduce investment to avoid the cost of carbon taxes. \citet{jacobzwerer2022investment} is the first to establish a negative link in a robust empirical study within the specific context of Spain.\footnote{
Previous research found limited evidence of changes in firms’ tangible assets or only modest reductions in energy consumption \citep{Martin2014,dussaux2020joint}. \citet{jacobzwerer2022investment} provide evidence of nearly a 1\% decrease in firm-level fixed tangible investments in response to the local emissions tax in the Valencia community of Spain. The authors argue that firms bear the environmental tax similarly to a corporate tax, with the tax rate $\tau_{e}$ being additive to $\tau_{c}$ in Eq. \ref{equ_theory_coc_maindraft}, so $\tau\prime_{c}$ can be interpreted as $\tau_{e} + \tau_{c}$. This finding is validated through a stacked difference-in-differences design using two CO2 tax introductions.}
Second, firms may adopt greener technologies to boost output while reducing emissions. However, studying this is challenging due to the often unobservable nature of these investment technologies.
\citet{brown2022canenvironmental} overcome this challenge by focusing on innovation metrics, showing that that a one standard deviation increase in sulfur oxide taxes raises firms’ R\&D expenses by 11\%. While no average effect on innovation outcomes was found, firms filed for more patents in air pollution abatement technologies, suggesting these taxes drive clean tech development in polluting industries.\footnote{\citet{colmer2024does} also document that French firms invested in pollution-reducing technologies in response to higher carbon prices, maintaining economic activity. \citet{Yamazaki2022} studied the link between carbon taxes and productivity, finding evidence of within-firm technological improvements following Canada’s 2008 carbon tax introduction.}
Given the early and setting-specific nature of the evidence on this important issue, there is clearly room for several significant studies on the mechanisms involved in firms' responses to environmental taxes.


%LEakage
Finally, firms may engage in carbon leakage to avoid environmental taxes, either in addition to or without necessarily reducing emissions or investing in greener technologies. Studies using Carbon Disclosure Project (CDP) data on multinational firms’ emissions by region provide inconclusive evidence on this issue \citep*{dechezlepretre2022searching,ben2021exporting}. \citet{kanzig2023leakage} leverage policy variation in Europe and provide robust large-sample evidence that multinational firms in heavy industries shift polluting activities to their African subsidiaries in response to higher carbon taxes in Europe.
However, the generalizability of these findings to other regions and settings remains uncertain. Carbon leakage poses a significant threat to the global goal of reducing emissions, and Europe’s Carbon Border Adjustment Mechanism (CBAM) aims to address this issue. Yet, CBAM is viewed as a costly policy in terms of compliance and administration. More evidence is needed to assess the impact of carbon leakage and the costs and benefits of tax and disclosure regimes like CBAM.



\subsubsection{Tax Strategies as an ESG Outcome}
%Question: Is tax avoidance ESG
A key question in the ESG space concerns the relationship between firms’ tax affairs and overall sustainability characteristics. So far, the literature has mostly examined correlations between proxies for tax avoidance and CSR measures, with mixed conclusions. Some studies suggest that corporate tax avoidance is linked to irresponsible corporate behavior \citep*{Hoi2013,lanis2015corporate,AlHadi2022,overesch2022relation,Watson2015}, while other research finds no evidence that tax avoidance is associated with firms’ broader social responsibility outcomes or reporting behavior \citep*{Davis2016,Mayberry2021,col2019going}.

The inconclusiveness of this body of work likely stems from the fact that, with the exception of \citet{Mayberry2021}, most studies lack exogenous variation for identification and struggle to identify a clear mechanism linking sustainability considerations to tax outcomes, or vice versa. This leaves important open questions, such as whether socially responsible firms pay higher taxes, whether tax-aggressive firms actively decouple tax from broader sustainability strategies, and what firm characteristics and economic consequences are associated with these tax-related ESG strategies.

%\citet{Mayberry2021} a quite narrow and specific setting, mechanism not entirely clear

\subsubsection{Tax-related Sustainability Disclosures and Real Effects}\label{section:sustainability_disclosures} %Ask elisa for feedback on this section

%MAndatory Regimes 
Some limited evidence suggests that broader, non-tax-specific sustainability disclosure mandates can influence firms’ real actions in ways that correlate with their tax strategies. 
\citet{Rauter2020} finds that the public CbCR mandate in extractive industries led to a 12\% increase in extraction payments and a nearly 5\% reduction in investment, suggesting higher costs and lower returns, consistent with investor reactions documented in \cite{johannesen2016power}. \citet{Rauter2020} also observes activity shifting to unregulated firms, reducing sector productivity. \citet*{fiechter2022real} show that EU firms increased CSR activities after the 2014 reporting mandate, which may also have influenced tax strategies, as tax controversies are often included in CSR ratings.
As sustainability disclosure regulations continue to be implemented globally, they are likely to impact firms’ reporting and real decisions (e.g., \citealp{fiechter2022real,krueger2023effects}, \citealp{abraham2023esg}). Future research should explore the effect of general disclosure mandates on tax outcomes to better understand how firms integrate tax affairs into their broader sustainability strategies. A significant empirical challenge, but also a valuable contribution, would be to determine how changes in tax strategies are linked to shifts in operating activities resulting from sustainability reporting mandates. Addressing this question will require leveraging institutional features across settings and comparing firms with differing tax incentives but similar exposure to sustainability disclosure mandates.

%Effect of public mandatory regimes (and related)
To date, the evidence on the effectiveness of public tax-specific disclosures on tax and sustainability outcomes is limited and inconclusive. Prior research suggests that public pressure and reputational risk are necessary conditions for these policies to be effective \citep{belnap2023effect,BOZANIC2017,Rauter2020,dyreng2016effect}.\footnote{However, little is known about how consumers perceive firms’ tax behavior beyond anecdotal evidence. \citet*{asay2024taxboycotts} suggest that consumers are relatively insensitive to negative tax information, consistent with the lack of evidence for reputational effects of tax avoidance. This aligns with the idea that hyper-rational consumers may prefer tax-avoiding corporations if they expect savings to be passed on through lower prices \citep*{dyreng2022tax,Gallemore2014}.} \citet{Hoopes2018} examine Australia’s public tax return disclosure for large firms, finding some consumer backlash against tax-avoiding firms but only small increases in tax payments, particularly for large non-listed firms. Similarly, \citet*{bilicka2023tax} and \citet{xia2023qualitative} find no reduction in tax avoidance for listed U.K. firms after a reform requiring public qualitative tax disclosures. However, \citet{bilicka2023tax} and \citet{Kays2022} show that firms affected by public tax disclosure mandates alter voluntary tax disclosures, which are often boilerplate and likely aimed at mitigating consumer backlash and public scrutiny.\footnote{Such strategic voluntary tax disclosures resemble the biased ESG disclosures of firms facing stakeholder pressure \citep*{kim2011strategic,gatti2019grey,abraham2023esg}. Future research in tax settings could inform the broader debate on greenwashing in response to voluntary and mandatory sustainability disclosures.}
While the evidence is limited, it suggests that country-specific mandatory public tax disclosure regimes do not necessarily lead to higher tax payments and operational changes as a form of social performance, but likely alter voluntary disclosures and impose compliance and proprietary costs on affected firms. Future research on the upcoming public CbCR regime in the EU, a salient and economically important setting, could provide more robust insights into this issue.

%Voluntary frameworks
Recent descriptive papers discuss how firms adapt to public tax-related sustainability disclosures and their potential consequences. While tax-related disclosures are required under mandates like the NFRD in Europe, they remain largely voluntary, as reporting frameworks like GRI allow firms to decide the materiality of topics \citep*{kopetzki2023moving}.\footnote{NFRD stands for the Non-Financial Reporting Directive, a European Union directive requiring large companies to disclose information regarding their social and environmental impact. GRI stands for the Global Reporting Initiative, an international independent organization that provides standards for sustainability reporting. GRI 207 specifically refers to the Tax 2019 standard, which provides guidance for reporting on tax-related issues, including CbCR.} As a result, tax-related disclosures vary widely in both quality and consistency. For example, 68\% of the 112 largest listed EU firms studied by \citet{kopetzki2023moving} did not consider taxes material for sustainability reporting, and 82\% provided only qualitative disclosures (see also \cite{hardeck2016taboo} and \cite{reiter2020tax} for similar findings).
\citet*{adams2022tax} find that firms with lower effective tax rates provide fewer voluntary tax disclosures and are less likely to offer informative CbCR disclosures. However, \citet*{Boer2024} show no significant relationship between tax aggressiveness and tax disclosures. Yet, this paper finds that firms more engaged in sustainability activities are more likely to make tax-related disclosures. This supports the idea that broader sustainability mandates may impact both firms’ tax affairs and their real effects -— an open question for future research.
Despite these findings, the availability and informativeness of public tax sustainability disclosures remain limited, leaving it unclear whether investors or stakeholders would act on this information, potentially influencing corporate behavior \citep{leuz2016economics}.


\subsubsection{Synthesis and Implications for Future Research}\label{section:greentaxes_synthesis}
%Environmental tax overview
Combating climate change is a top priority for global policymakers, and environmental taxes are widely viewed as a critical tool for reducing greenhouse gas emissions (e.g., \citealp{acemoglu2016transition,dimauro2021combatting}). Accounting scholars have significant opportunities to explore firms’ responses to environmental taxes. The evidence across disciplines is still young and inconclusive due to the recent implementation of effective policies and challenges in measuring firm-level responses.
However, our analysis of U.S. Form 10-K reports from 2006-2022 in Figure \ref{fig:greentaxes_10ks}  reveals that an increasing number of firms are mentioning environmental taxes, suggesting  opportunities for further research exploiting the growing financial materiality as shock to firms' cost considerations and the available information disclosed by firms as data sources.
We believe accounting researchers are well-positioned to address many important yet unanswered questions about how firms respond to environmental taxes. 


\begin{figure}[ht!] 
\caption{Trend in the Materiality of Environmental Taxes for U.S. Firms }  \label{fig:greentaxes_10ks}
    \centering
   \includegraphics[width=0.95\linewidth]{form_mentions_percentage.pdf}
   \subcaption*{\textit{Notes:} This figure shows the annual share of Form 10-K annual reports published by U.S. firms that mention carbon price policies (in \%). We separately show mentions of cap-and-trade systems (left blue bars) and environmental taxes (right pink bars). The sample includes the universe of Form 10-Ks available on Edgar.}
\end{figure}


%Environmental tax: specific suggestions
First, while \citet{martinsson2024effect} and \citet{erbertseder2023effective} offer new evidence into emissions reductions in specific settings, more studies on this first-order policy question could explore firm-level responses across different jurisdictions and firm characteristics. By leveraging public and private disclosures, accounting researchers can better measure emissions and firms’ exposure to environmental policies, distinguishing between reporting, avoidance, and real responses.  
At this point, researchers should explore firm responses to broader policies beyond carbon taxes, such as carbon trading schemes, subsidy regulations like the U.S. Inflation Reduction Act (IRA), and taxes aimed at promoting circular economies (e.g., plastic taxes), deforestation prevention, water cleanliness, or consumer-targeted incentives for electric vehicle purchases \citep{giese2023towards}.
Second, such granular measurement will also help reconciling conflicting evidence on the impact of environmental taxes on a broader set of firm responses. For example, studies like \citet{brown2022canenvironmental} highlight increased green technology investments, while \citet{jacobzwerer2022investment} find reduced tangible investments in response to larger environmental tax burdens. Future research can also link innovation outcomes reported in firms’ disclosures \citep{glaeser2024review} to environmental taxes to examine how these policies drive technological change.
Third, research could address the economic costs of climate-related tax policies, such as compliance costs, resource allocation,  competitive distortions, all of which can inhibit growth and equity.  Fourth, research is needed to reconcile firm-level findings with macroeconomic evidence is crucial, as environmental taxes may reduce emissions through various mechanisms: reduced investment, shifts to greener business models, or increased investment in green technologies. \citet{jacobzwerer2022investment} offer intriguing insights on lower firm-level investment, but when paired with the lack of macroeconomic evidence on negative GDP and employment effects in \citet{metcalf2022macroeconomic}, it leaves open questions about sector-specific responses, geographic reallocation, and business model changes.

%Additional areas
Three additional areas are particularly relevant, as accounting scholars can leverage their expertise in measuring firms’ tax incentives and how they communicate environmental risks and benefits. 
First, accounting researchers can help evaluate and compare different measures of firm-level emissions and emission intensities, an area lacking consensus on measurement \citep*{aswani2024carbon}. %While climate scientests \hl{talk abouto their comparative advantage and then accountiing}
  By leveraging firm-level disclosures and financial statement figures, future work can provide insights into the extent and nature of emissions in relation to a business’s operations. Thus, accounting researchers have the potential to contribute to a more coherent cross-disciplinary standard for measuring outcomes of interest, a curent issue in the investment literature (see Section \ref{section:synthesisinvestment}).
Second, researchers can examine whether environmental tax incentives are more effective under transparent reporting regimes and whether firms engage in greenwashing or strategic reporting instead of making real changes. Third, the issue of carbon leakage is akin to corporate income tax avoidance, a major area of expertise in accounting research \citep{wilde2018perspectives}. Thus, this area offers potential for studies on firms’ reporting of environmental tax bases versus real activities and the role of environmental tax havens \citep{levinson2008pollution}. 

%ESG Synthesis:
Beyond the impact of environmental tax policies on firm behavior, the role of taxation in firms’ sustainability outcomes is an emerging and promising area for tax accounting research. Several studies have explored whether corporate tax avoidance is linked to irresponsible corporate behavior, but inconclusive  evidence leaves this question  unresolved. 
ESG strategies can lead to technological shifts, changes in workforce composition, and adjustments to supply chains, all of which have tax implications. These changes may affect access to R\&D credits, payroll taxes, and the allocation of taxable income across jurisdictions, particularly through transfer pricing. As firms increasingly view ESG as a value-creation factor \citep{servaes2013impact,hawn2016mind,lins2017social}, future research could also investigate the international tax and transfer pricing consequences and firms strategies  where to locate ESG-related assets and functions.

%Future Research Directions:
Future research could particularly improve our understanding by focusing on exogenous shocks to tax avoidance opportunities or exposure to ESG disclosure regulations and ESG-focused investors. Promising settings include the EU’s mandatory public CbCR mandate, the adoption of GRI 207 tax reporting standards, and firms’ commitments to ESG initiatives like the United Nations Principles of Responsible Investing (UN PRI) or the Carbon Disclosure Project (CDP) \citep{kim2023analyzing,cohen2023institutional}. Additionally, studying the effect of existing tax rules on ESG outcomes where there is a clear theoretical link, as seen in \citet{Yost2022}, which finds stricter tax enforcement reduces insider trading and stock gift manipulation, could yield valuable insights.
To link predictions to the empirical design as directly as possible, future work would benefit from using alternative data sources and developing relevant measures of tax-related sustainability disclosures, as it remains unclear whether stakeholders (e.g., consumers or the media) engage with the fragmented landscape of current tax-related reports. Potential data sources include firm-specific news \citep{li2023retail}, company websites \citep{abraham2023esg}, and social media \citep{gomez2021stakeholders}.


% Settings and data accounting researchers can exploit: ETS in Europe works like carbon tax and continues to provide time-series shocks \cite{kanzig2022unequal}. Recently,  energy tax credits have been implemented. Interesting  Datasets: \cite{janssens2013global,edgar_data}. Sattelite data. Data from US authority EPA TRI: \cite{Hanna2010} and \cite*{abraham2022esg}.


 
%%%%%%%%%%%%%%%%%%%%%%%%%%%%%%%%%%%%%%%%%%%%%%%%%%%%%%%%%%%%%%%%%%%%%%%%%%%%%%%%%%%%%%%%%%%%%%%%%%%%%%%%%%%%%%%%%%%%%
%DISCLOSURE
%%%%%%%%%%%%%%%%%%%%%%%%%%%%%%%%%%%%%%%%%%%%%%%%%%%%%%%%%%%%%%%%%%%%%%%%%%%%%%%%%%%%%%%%%%%%%%%%%%%%%%%%%%%%%%%%%%%%%
\subsection{Real Responses to Tax Disclosure Regulation} \label{section:disclosureregulation}
\subsubsection{Theoretical Underpinnings and Overview}
Regulators require firms to disclose tax information,  either publicly or privately to tax authorities, typically to curb aggressive tax behavior \citep{hoopes2022taxdisclosure,olbertspengel2023}. These mandates aim to improve tax authorities’ information and enable better enforcement. As discussed in Section \ref{section:framework_extended}, the investment effects of increased enforcement are ambiguous: stronger enforcement could induce higher after-tax costs, leading firms to cut investment, or increased transparency from disclosure may improve institutional quality and certainty, leading firms to invest more. In the case of public disclosure mandates, firms may adjust their tax strategies and other disclosed activities in anticipation of public scrutiny (see Section \ref{section:sustainability_disclosures}).
In their review, \citet{hoopes2022taxdisclosure} summarize evidence that firms adjust financial reporting to avoid additional disclosures to tax authorities (e.g., \citealp{abernathy2013schedule,Towery2017,Honaker2017,BOZANIC2017,belnap2023effect,xia2023qualitative}). Furthermore, prior research in the broader accounting field documents real effects of voluntary disclosure mandates \citep*{roychowdhury2019effects}. We review the emerging empirical literature examining firms’ real responses to tax disclosure mandates in both international and U.S. contexts.
 


\subsubsection{International Tax Disclosure Regulation}
%Intro
In the past two decades, governments worldwide have introduced coordinated mandatory transparency rules and information exchange agreements (see \cite{olbertspengel2023} for a review). The primary goal is to help tax authorities audit the operations and tax positions of foreign multinationals more effectively. The most significant recent international disclosure regulations are the private and public Country-by-Country Reporting (CbCR) regimes. These regimes provide tax authorities in multiple jurisdictions with new information on firms’ operations, potentially affecting not only overall investment levels in response to tax incentives but also the global allocation of resources. Currently, private CbCR applies to multinationals with consolidated revenues exceeding EUR 750 million and operations in at least one country participating in the OECD minimum standard. %\footnote{Besides the internationally coordinated rules, some domestic policymakers have introduced public tax disclosure mandates with a focus on multinational firms' cross-border operations. For example, the Australian tax office has published a single-country version of a CbC report (i.e., based on Australian operations) for large firms since 2013. } 

%Private CbCR and real effects
A growing body of accounting literature provides evidence on the real effects of private CbCR, building on the extensive research on the real effects of mandatory disclosures in accounting \citep{kanodia2016real,leuz2016economics,roychowdhury2019effects}.\footnote{We review studies on emerging \textit{public} CbCR mandates in Section \ref{section:sustainability_disclosures}, as these rules are expected to elicit pressure from various stakeholders, which could prompt firms to alter their broader sustainability practices.}
\citet{desimone2022realeffects} reconcile subsidiary-level unconsolidated data with firm-level consolidated accounts and ownership information. Using both a regression discontinuity and a difference-in-differences design, they find no changes in firms’ consolidated human capital or physical investments. However, they document significant reallocation effects, with investment shifting to tax-favorable European jurisdictions and away from tax havens. This suggests that multinationals are substantiating tax-motivated transfer pricing structures by reporting large tax bases in countries like Luxembourg, the Netherlands, or Ireland.
Overall, these findings indicate that affected multinationals made real operational changes, rather than simply altering tax reporting strategies, which aligns with the small effects on ETRs and the limited evidence of reduced profit shifting in \citet{Joshi2020} and \citet{hugger2024impact}.


%CbCR and misalignment/follow up studies
Given the global scale of the CbCR regime and the complexity of multinational operations, several open questions remain. Key issues include the nature of firms’ reallocation activities, the magnitude of spillover effects across countries or business units, the aggregate impact on individual jurisdictions, and implications for regions beyond Europe.
\citet*{joshi2023private} and \citet{Afrin2024} offer recent insights. \citet{joshi2023private} find that multinationals reduce the misalignment between reported profits and real activities in high-tax countries. Their macroeconomic data analysis suggests that countries with previously low profits-to-assets ratios benefited most from reallocation. \citet{Afrin2024} provide early evidence that investment reallocation may be driven by M\&A activity. While these studies contribute to understanding CbCR’s effects, their results likely capture a mix of firm responses to CbCR and correlated factors, and they do not examine regions outside developed high-tax countries or investment activities beyond those observable in commercial databases.

%I.e., in high-tax countries, multinationals increase either the level of fixed assets if reported profits were high prior to CbCR, or they reduce reported profits if the firm allocated a lot of fixed assets to that country prior to CbCR. 
%However, the study does not investigate multinational firms' operations in tax-favorable jurisdictions or consolidated firm investment.

  %Disclosures about foreign activities and foreign earnings/locked out cash: 
%\cite*{blouin2018location}
\subsubsection{Evidence from the U.S.} 
Several  studies provide evidence that additional disclosures made to U.S. tax authorities decrease or delay investments.  For example, disclosure of uncertain tax positions, through Schedule UTP or FASB FIN 48, decrease or delay investments, likely because managers anticipate greater tax uncertainty and higher future tax payments \citep{jacobwentland2022real,goldman2022fasb,goldman2023noninnovative}. An alternative or additional mechanism explaining these investment effects could be proprietary costs. For example, \citet{Yost2023} finds that tax-aggressive firms are more likely to de-list before the adoption of FIN 48, suggesting that mandatory disclosure rules like FIN 48 impose tax-related proprietary costs, potentially reducing future investment by limiting access to capital markets \citep{bharath2014going,dobridge2024ipos}.\footnote{Schedule UTP is a U.S. tax form where corporations disclose tax positions for which uncertain liabilities have been recognized in the financial statements.  FASB FIN 48 (now part of ASC 740) is an accounting standard requiring companies to recognize and disclose uncertainties in their tax positions on financial statements. They are related in that Schedule UTP reports to the IRS the same uncertain tax positions that FIN 48 requires to be disclosed in financial reporting. Beyond the proprietary cost channel, prior research suggests that both standards likely reduced the perceived quality of financial accounting information by capital market participants (e.g., \citealp{Towery2017, Robinson2016}), potentially influencing firms’ investment decisions through capital market effects. }
Future research could explore the relative importance of these mechanisms, which may depend on firms’ capital market pressures and could be studied in non-U.S. settings. As noted in Section \ref{section:capex}, studies in this area should also carefully benchmark disclosure effect sizes against the impact that the underlying tax benefit (or cost) would have on investment outcomes.



\subsubsection{Regulatory Avoidance and Real Effects} %synthezied down to one paragraph
Given that disclosure mandates typically impose costs on firms \citep{leuz2016economics}, some firms actively seek to avoid tax transparency mandates via reporting choices. Studies show that firms, particularly private firms with less capital market pressure, under-report pre-tax income \citep{hasegawa2013effect,Hoopes2018} or revenue in the case of CbCR \citep{hugger2024impact}, to avoid hitting disclosure thresholds. These findings suggest firms are willing to misrepresent their performance to reduce expected tax payments or prevent additional scrutiny from global tax authorities. While this behavior highlights regulatory arbitrage, it can also have significant real effects, as shown in other contexts of accounting and disclosure regulation, such as size management to avoid public financial reporting mandates in Europe \citep{bernard2018size} or size-dependent tax treatments as in the case of banks' loan loss provisioning \citep{Andries2017}. Important open questions remain regarding the potential unintended effects of size-based mandatory reporting thresholds on firms' investment and organizational structure, as well as the reporting responses that firms engage in to avoid such thresholds.  

%Details on studies, MUTED for streamlining
%\cite{hasegawa2013effect} study a domestic tax information disclosure regime in the early 2000s in Japan and estimate that approximately 5-8\% of firms above the disclosure threshold) under-report their true income to avoid disclosure. The findings in \cite{hasegawa2013effect}  suggest that up to 2,500 Japanese firms with pre-tax income figures of approximately USD 350,000 are willing to artificially reduce their reported income by an amount sufficiently large to keep the firms under the disclosure threshold. Thus,  in this setting, the behavioral responses were driven by a large population of small firms. 
%\cite*{Hoopes2018} find consistent evidence when exploiting a similar regulation in Australia which affected large firms. In  2015, the Australian Taxation Office started to publicly disclose taxable income and tax payments  from tax returns of approximately 1,500 firms with taxable income of at least AUD 100 million. The authors document a significant increase in the excess mass of  particularly private and foreign-owned firms just under this reporting threshold when the disclosure rule was announced.
% \cite{hugger2024impact} provides early evidence that such responses can also be expected in the international CbCR setting. The evidence based on a host of tests  isconsistent with firms actively decreasing their consolidated revenues reported in the financial statements to fall under the EUR 750 million size threshold of mandatory private CbCR. 


\subsubsection{Synthesis and Suggestions for Future Research}
%Disclosure regulation in general
Emerging research on tax disclosure mandates in the U.S. has documented significant costs, including reduced and less risky investments. However, evidence from other settings remains limited, leaving room for further investigation into the role of tax disclosures in firms' real decisions. In particular, there is little evidence on the role of information intermediaries (e.g., \citealp{edwards2024third}), as well as limited evidence on potential capital market effects \footnote{\cite*{lennox2015tax} provide indirect evidence showing that domestic disclosure requirements for R\&D investments reduce R\&D investments in tax-aggressive firms, as these firms avoid scrutiny by tax authorities.} 
One specific angle worth investigating is the potentially different effects on private versus publicly listed firms. While both must report tax information to tax authorities in the same way, private firms face significantly fewer financial statement disclosure mandates as well as capital market pressures compared to public firms, and thus the economic impact may manifest in smaller or different ways. Finally, future research should also reconcile the magnitudes of firm responses across studies and compare effect sizes in tax disclosure to those found in non-tax disclosure settings, a central area of accounting research. When pursuing these questions, researchers should aim to improve upon the existing work by addressing identification challenges arising from endogenous policy choices and the simultaneity of economic and disclosure changes, along with accurately measuring the extent to which firms are affected by these mandates.

%Disclosure regulation international
In international settings, research on private CbCR has started to investigate various real effects. 
As CbCR is a landmark regulation in the international context, further research is needed to examine both the real and reporting effects of CbCR across diverse tax systems, macroeconomic conditions, and information environments. To make meaningful contributions, future studies should clearly articulate the theoretical mechanisms through which CbCR influences the outcomes of interest and design empirical tests that effectively isolate these mechanisms. Given that CbCR represents a significant policy shock—coinciding with other tax reforms and affecting multiple firm outcomes—such an approach is essential to disentangle its effects and account for the interdependence of firm decisions. Targeted research of this kind will also help reconcile existing findings, particularly regarding their economic magnitudes in comparison to other disclosure studies.
Additionally, it remains unclear whether the documented effects stem from disclosing CbCR reports to one domestic tax authority or from the information sharing across jurisdictions, as mandated by the EU from 2024. Surprisingly, the impact of third-party reporting and information sharing through the CbC Multilateral Competent Authority Agreement has not been thoroughly examined, despite evidence from similar regimes showing significant effects on transparency and business practices.\footnote{See, e.g., \cite{Bennedsen2018,oDonovan2019value,nesbitt2023reexamination,Li2022,brown2019interplay,eberhartinger2021banks,chow2023cross}; see also \cite{olbertspengel2023} for an overview of this work. Studies on the OECD Common Reporting Standard (CRS) and the U.S. FATCA \citep*{desimone2020fatca,Casi2020} show reductions in tax haven use and avoidance by individuals. As firms often access financing through tax havens, regimes like CRS and FATCA could also have broader implications for firms due to the reallocation of financial capital.} 
Another avenue requiring more attention in future research is the broader impact of international tax transparency regulations initiated by developed countries with strong information and tax enforcement systems. Specifically, developing countries, where tax authorities have historically had limited access to information about large MNCs’ global operations and tax affairs, may experience different effects. The predictions here are unclear: additional information could increase tax risk and the potential for expropriation by captured authorities. However, multilateral information exchange agreements often lead to improvements in institutional quality within participating countries.




%Tax enforcement and real effects Information Sharing
More broadly, there is limited evidence on firms’ real responses to tax enforcement, often driven by tax disclosure mandates. Recent studies suggest that heightened tax enforcement negatively impacts firm survival and business activity \citep*{belnap2022real,gallemore2025corporate}. At the same time, greater enforcement can also improve the reliability of tax return information, which in turn could have positive economic effects through its use in contracting (e.g., \citealp{GallemoreJacob2020}). Given the increasing prevalence of tax enforcement, budget constraints for funding enforcement, and the potential of digital technologies to enhance tax oversight, further research can explore the specific and to-date under-explored mechanisms through which tax disclosures drive economic outcomes.

%Finally, a promising avenue for future research concerns tax-related compliance costs and how this may affect firms' real and reporting responses to tax policy changes.  The regulatory trend towards additional tax-related disclosures increase compliance costs, and the relevance of these costs likely varies by firm size, exposure to tax incentives and burdens, existing tax planning stratagies and accounting systems, and more. To date, the literature on these costs is relatively limited.  This limitation is primarily due to data restrictions, with researchers using data such as that from Auditor-Provided Tax Services.  However, there are likely selection considerations about those firms that use their auditor for tax consulting and compliance, and firms may employ different advisors for different projects based on firm expertise or client-firm relationships.  In light of the new and additional disclosure mandates that also likely generate new data points, impactful future work can possibly even tries to leverage relationships between accounting scholars and accounting firms to better understand the amount of costs and the nature of services provided.  
Finally, a promising  research area are tax-related compliance costs and their direct and indirect impacts on firms’ real and reporting responses to tax policy changes. The global trend toward increased tax-related disclosures raises compliance costs, which likely vary based on firm size, exposure to tax incentives and burdens, existing tax planning strategies, accounting systems, and other factors \citep{leuz2016economics,Ayers2015}. Despite their importance, research in tax on these costs remains relatively limited, primarily due to data constraints. Existing studies often rely on Auditor-Provided Tax Services data, but this approach has selection concerns, as firms that use their auditor for tax consulting and compliance may differ systematically from others. With new disclosure mandates generating additional data points, future research could benefit from leveraging relationships between accounting scholars and accounting firms to better quantify compliance costs and examine their relations with reporting decisions and potentially indirect knock-on effects on real outcomes due to stakeholders' reactions to the reported information, which in turn affects firms' investment and other outcomes.

%%%%%%%%%%%%%%%%%%%%%%%%%%%%%%%%%%%%%%%%%%%%%%%%%%%%%%%%%%%%%%%%%%%%%%%%%%%%%%%%%%%%%%%%%%%%%%%%%%%%%%%%%%%%%%%%%%%%%%%%%%%%%%%%%%%%%%%%%%%%%%%%%%%%%%%%
%%%%%%%%%%%%%%%%%%%%%%%%%%%%%%%%% Section Conclusion %%%%%%%%%%%%%%%%%%%%%%%%%%%%%%%%%%%%%%%%%%%%%%%%%%%%%%%%%%%%%%%%%%%%%%%%%%%%%%%%%%%%%%%%%%%%%%%%%%%%%%%%%%%%%%%%%%%%%%%%%%%%%%%%%%%%%%%%%%%%%%%%%%%%%%%%%%%%%%%%%%%%%%%%%%%% 
\section{Concluding Discussion}\label{section:conclusion}
This manuscript reviews empirical research on the real effects of taxation, with a particular focus on recent contributions from accounting scholars. Our review of the literature is grounded in canonical theory \citep{hall1967tax}, with corresponding extensions that help researchers incorporate firms' incentives and constraints not modeled in the simple neoclassical investment case \citep{scholeswolfson1992}. We then propose an organizing framework that encompasses the wide range of outcomes and tax policies studied by tax accounting scholars.


Tax accounting research makes four key contributions to the study of real effects. First, it demonstrates that real responses to tax policy vary based on financial reporting incentives. Second, it shows that firms often also engage in pure reporting adjustments instead of or in addition to real spending in response to tax incentives. Third, it reveals (at times unintended) real effects of mandatory tax disclosure regimes. Fourth, it improves measurement of firms’ tax status and real and reporting responses.

%Conclusion paragraph on research design. \hl{New - research design discussion in two paragraphs instead of entire section, points from MB to be included}
Accounting research increasingly focuses on causal relationships to evaluate tax policies and estimate elasticities (e.g., in \citealp{Harris2018,Joshi2020,desimone2022realeffects,gallemore2024tax,belnap2022real}). Such work has the potential to produce novel insights and inform policymakers and scholars across various disciplines \citep{breuerdehaan2024using,leuz2022towards}, particularly when researchers use robust empirical methods for credible identification and carefully interpret the exploited variations.\footnote{We stress that not all tax research requires strict causal identification, but it is crucial to transparently address identification challenges and clarify whether a study aims to provide descriptive insights or establish causality. Descriptive studies, especially those exploring understudied topics, offer substantial value by uncovering novel facts through thorough analysis of correlations, trends, and patterns in high-quality datasets (see \citealp{armstrong2022causality} for a discussion). While tax research often revisits existing questions due to the stable nature of tax systems, refining earlier work with stronger research designs can yield valuable policy-relevant insights. We hope reviewers recognize the significance of such work on fundamental questions, even if it does not overturn existing results.}
%However, the mere accumulation of studies in one area does not help fully answer a policy question if existing research still suffers from identification and measurement problems \citep{leuz2022towards}.
%Thus, we encourage researchers to improve upon existing work in terms of data and identification if the work tackles first-order questions. 
%\footnote{For example, the evidence in \cite{Rao2015}  on employment reallocations after corporate inversions is likely not causal but provides novel and striking descriptive insights that deserve further attention to draw strong conclusions and inform policymakers and other stakeholders.}
%Our literature review highlights instances where the current research has not fully utilized accounting scholars’ institutional and measurement expertise and makes suggestions how to develop improved  empirical strategies. 

Beyond encouraging the adoption of state-of-the-art empirical tools (see \citealp{doerrenbergolbert2025empirical} for an overview), we conclude with three general methodological recommendations to enhance the potential of the field.
First, adopting a design-based approach as outlined in \citet{leuz2022towards}, coupled with clear theoretical predictions, is particularly beneficial for tax accounting research given the complexity of informational and institutional nuances.\footnote{For example, when studying the effects of public Country-by-Country reporting, it is crucial not only to develop an identification strategy that leverages plausibly exogenous variation but also to clearly articulate the remaining useful and potentially problematic variation, such as differences between listed and private firms or those facing different regulators (see \citet{Bethmann2018} for a detailed discussion on managing residual variation). A good example of a clear theory to test is provided in \citet{reineke2021transfer} and  \citet{reineke2023}, which predict how anti-tax avoidance measures affect firms’ investment levels and how international tax planning relates to investment efficiency.}
Second, we recommend triangulating main results by using alternative data sources and combining narrow, high internal validity empirical strategies with broader designs that may have less stringent controls. This approach, as suggested by \citet{armstrong2022causality}, is particularly useful in tax policy settings where clearly exogenous treatment variation is often absent. %\footnote{Recent examples include \citet{belnap2023effect}, who combine field experiments with survey data, and \citet{Langenmayr2018}, who use both a narrow, single-country regression discontinuity design and a cross-sectional international sample to enhance internal and external validity.}
Third, we encourage greater use of visualizations in tax research. Visualizing patterns in raw data,  illustrating identification strategies and institutional settings, and plotting rather than tabulating DiD estimates can enhance the credibility and accessibility of results. 

 Throughout our review, we identify several open research questions and make suggestions for future research. Key areas include reconciling conflicting findings or diverging economic magnitudes of effect sizes, better measuring incremental investment responses to tax policies, and disentangling the relationship between tax planning and real firm activities, in particular cross-border profit shifting. We also emphasize the importance of an interdisciplinary approach, integrating insights, data, and methodologies from public economics as well as accounting disclosure theory to better understand the role of taxation in firm behavior. Additionally, new global policy developments, such as the global minimum tax and carbon border adjustment mechanisms, provide laboratories for future research to examine the mechanisms driving firm responses and the effectiveness of disclosure regimes. The research on the real effects of taxation has ample potential to significantly advance both academic understanding and policy relevance in the field of tax accounting.






\bibliography{References}

\input{Online_Appendix_Elsevier.tex}

\end{document}
