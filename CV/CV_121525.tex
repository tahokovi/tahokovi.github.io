%%%%%%%%%%%%%%%%%%%%%%%%%%%%%%%%%%%%%%%%%
% Medium Length Professional CV
% LaTeX Template
% Version 3.0 (December 17, 2022)
%
% This template originates from:
% https://www.LaTeXTemplates.com
%
% Author:
% Vel (vel@latextemplates.com)
%
% Original author:
% Trey Hunner (http://www.treyhunner.com/)
%
% License:
% CC BY-NC-SA 4.0 (https://creativecommons.org/licenses/by-nc-sa/4.0/)
%
%%%%%%%%%%%%%%%%%%%%%%%%%%%%%%%%%%%%%%%%%

%----------------------------------------------------------------------------------------
%	PACKAGES AND OTHER DOCUMENT CONFIGURATIONS
%----------------------------------------------------------------------------------------

\documentclass[
	%a4paper, % Uncomment for A4 paper size (default is US letter)
	9pt, % Default font size, can use 10pt, 11pt or 12pt
]{resume} % Use the resume class

\usepackage{ebgaramond} % Use the EB Garamond font
\usepackage[colorlinks=true, urlcolor=blue]{hyperref}
\usepackage{graphicx}
\usepackage[absolute,overlay]{textpos}


%------------------------------------------------

\name{Tonga M. Jensen Ahokovi} % Your name to appear at the top

% You can use the \address command up to 3 times for 3 different addresses or pieces of contact information
% Any new lines (\\) you use in the \address commands will be converted to symbols, so each address will appear as a single line.

\address{Stanford Institute for Economic Policy Research (SIEPR) \\ Stanford, CA 94305-6015} % Main address

%\address{123 Pleasant Lane \\ City, State 12345} % A secondary address (optional)

\address{(808)~$\cdot$~517~$\cdot$~7704 \\ tahokovi@stanford.edu \\ tonga.ahokovi@gmail.com \\ \href{https://tahokovi.github.io}{[My Website]}} % Contact information

%----------------------------------------------------------------------------------------

\begin{document}


% First image block
%\begin{textblock*}{4cm}(18cm,0.5cm) % {block width} (coords)
%\includegraphics[width=1.75cm]{aei.jpeg}
%\end{textblock*}

% Second image block
%\begin{textblock*}{4cm}(18cm,2.25cm) % Adjust coordinates as needed
%\includegraphics[width=1.75cm]{UHlogo.jpg}
%\end{textblock*}



%----------------------------------------------------------------------------------------
%	EDUCATION SECTION
%----------------------------------------------------------------------------------------

\begin{rSection}{Education}

\textbf{$\cdot$ Stanford University}, Palo Alto, CA \hfill \textit{July '25 -- present}
\begin{list}{}{\leftmargin=1.5em \itemsep=0pt \parsep=0pt \topsep=0pt}
\item[] Non-matriculated Graduate Student \smallskip \\
\begin{tabular}{@{}l @{\hspace{6ex}} l @{}}
    Research Interests: & Applied Microeconomics, Machine Learning Applications in Economics,
                         Labor Economics, \\ & Urban Economics, Public Economics, Economic History.\\
\end{tabular} \\
\rule{\linewidth}{0.4pt} \\
\begin{tabular}{@{}l @{\hspace{6ex}} l @{}}
    \textbf{Relevant Coursework:} Topics in Economic History (PhD), Proofs \& Modern Mathematics.
\end{tabular}
\end{list}

\textbf{$\cdot$ University of Hawaii at M\=anoa}, Honolulu, HI \hfill \textit{2021-2024}
\begin{list}{}{\leftmargin=1.5em \itemsep=0pt \parsep=0pt \topsep=0pt}
\item[] B.A. Economics (\textit{cum laude}) -- Quantitative Economics Concentration \smallskip \\
\begin{tabular}{@{}l @{\hspace{6ex}} l @{}}
    \textbf{Relevant Coursework:} & \\
    Mathematics: & Differential, Integral, and Multivariable Calculus (Calculus I-III), Linear Algebra, \\
                 & Advanced Mathematics (methods of proof, naive set theory, axiomatic systems). \\
    Economics: & Econometrics I (PhD), Econometrics I-II, Mathematical Economics \\
               & (linear programming \& optimization), Intermediate Micro- \& Macroeconomic Theory, \\
               & Advanced Directed Research w/ Prof. Dylan Moore. \\
    Statistics \& Data Science: & Machine Learning Methods, Intro. Statistics, Data Analysis \& Visualization. \\
    \hline
    \textbf{Honors \& Awards:} & Member, Omicron Delta Epsilon Economics Honor Society, \\
                               & Member, Pi Gamma Mu International Honor Society in Social Sciences, \\
                               & Dean's List, College of the Social Sciences, \\
                               & HEA Scholarship Recipient, Hawaii Economic Association, \\
                               & Pipeline Scholar, American Real Estate \& Urban Economics Association. \\
                               & Don Lavoie Fellow (Spring '25), Mercatus Center \\
    \hline
    \textbf{Professional Affiliations} & Student Member, National Association for Business Economics, \\
                                       & Student Member, Hawaii Economic Association, \\
                                       & Founding Member \& Director of Professional Development, M\=anoa Economics Association.
\end{tabular}
\end{list}

\textbf{$\cdot$ Iolani School}, Honolulu, HI \hfill \textit{2015-2021}
\begin{list}{}{\leftmargin=1.5em \itemsep=0pt \parsep=0pt \topsep=0pt}
\item[] H.S. Diploma
\end{list}
\end{rSection}


%----------------------------------------------------------------------------------------
%	TECHNICAL STRENGTHS SECTION
%----------------------------------------------------------------------------------------

\begin{rSection}{Technical Skills}

	\begin{tabular}{@{} >{\bfseries}l @{\hspace{6ex}} l @{}}
		Computer Languages & R, Stata, and Python.  \\
            Packages \& Libraries & ggplot2, dplyr, tidyverse, tidyr, tidycensus, sandwich, plm, zoo, AER, data.table, \\
                     & NumPy, pandas, Matplotlib, SciPy, scikit-learn, Keras, Jupyter Notebook. \\
		Software & \LaTeX, ArcGIS, Microsoft Office Suite (Excel, Word, \& PowerPoint). \\
		Languages & English (Fluent), French, and Tongan (Conversational).
	\end{tabular}

\end{rSection}

\begin{rSection}{Working Papers}
    {\textit{\textbf{The Costs of Mandates that Waste US International Emergency Food Aid Resources: New Evidence}} \href{https://www.aei.org/wp-content/uploads/2025/12/The-Costs-of-Mandates-that-Waste-US-4.pdf?x85095}{[PDF]} \\ (with Vincent H. Smith and Philip G. Hoxie)  \hfill \textit{December 2025} \\ 
    AEI Economics Working Paper 2025-08. \smallskip\\
    \textit{Abstract:} This paper estimates the impact of cargo preference and US sourcing mandates on the freight costs incurred under the US international emergency food aid program between 2013 and 2024, extending previous work from Hoxie, Mercier, and Smith (2022). Compared to food aid shipments carried on foreign flagged vessels, on average the cargo preference requirement is estimated to increase ocean transportation freight rates substantially by \$77 per ton for packaged goods shipments and by \$84 per ton for bulk goods shipments. The evidence indicates that cargo preference freight rates continued to command sizeable premiums amounting to about 50 percent of the rates charged for packaged aid and nearly 100 percent of the rates charged for bulk grain shipped.}
    
    \newpage
    
    {\textit{\textbf{Taxation-Induced Tenancy: Evidence from Washington D.C.}} [PDF coming soon] \\ (Sole author)  \hfill \textit{July 2024} \\ 
    Accepted conference paper at the \textit{'24 National Tax Association Annual Conference on Taxation}, Detroit, MI. \\ \smallskip
    Presented at the \textit{'24 North American Meeting of the Urban Economics Association}, Georgetown University, DC. \smallskip\\
    \textit{Abstract:} Vacancy taxation is gaining popularity as a policy tool to address affordable housing shortages and promote urban revitalization. Despite this growing interest, there is a notable gap in the literature assessing the effectiveness of the policy. This paper is the first to examine the effects of vacancy taxation in the context of the United States, and capitalizes on a significant policy change -- a major property tax reform in Washington D.C. -- as a natural experiment. Employing a combination of standard and novel synthetic event study designs, I find evidence that the policy precipitated a large reduction in residential vacancies and an increase in occupied housing units.
    
    \textit{\textbf{The Supply Effects of Rent Control: New Evidence from New York}} [PDF coming soon] \\ (Sole author) \hfill \textit{October 2023} \smallskip \\ 
    \textit{Abstract:} This paper employs a synthetic difference-in-differences design to investigate the impact of a major policy change in New York's rent control law, as measured by the number of new private residential housing units authorized by permits. Contrary to arguments in favor of rent control as a mechanism to mitigate rising housing costs, I find a significant decrease in the number of housing units authorized following the policy change under study. This reduction is most pronounced in the multifamily housing sector.}
\end{rSection}

\begin{rSection}{Research Contributions} 

    This section features research projects where I contributed as a research assistant. My involvement included collaborating on research design, managing and cleaning data, performing analyses, applying statistical and econometric methods, and assisting with copy editing. \smallskip

    {\textit{\textbf{Aid for Incumbents: The Electoral Consequences of COVID-19 Relief}} \href{https://www.nber.org/papers/w32962}{[NBER WP]}  \hfill \textit{September 2024}} 
    \\ 
    Jeffrey Clemens, Julia Payson, and Stan Veuger
    \smallskip \\
    \textit{Abstract:} The COVID-19 pandemic led to unprecedented levels of federal aid transfers to state governments. Did this funding increase benefit state incumbents electorally? Identifying the effect of revenue windfalls on economic voting is challenging because whatever conditions led to the influx of cash might also benefit or harm incumbent politicians for a variety of other reasons. We develop an instrument that allows us to predict allocations to states based on variation in congressional representation. We find that incumbents in state-wide races in 2020, 2021, and 2022 performed significantly better in states that received more relief funding due to their over-representation in Congress. These results are robust across specifications and after adjusting for a variety of economic and political controls. We consistently find that the pandemic-period electoral advantage of incumbent politicians in states receiving more aid substantially exceeds the more modest advantage these politicians enjoyed during pre-pandemic elections. This paper contributes to our understanding of economic voting and the incumbency advantage during times of crisis as well as the downstream electoral consequences of both the COVID-19 pandemic and of unequal political representation at the federal level.
\end{rSection}

%----------------------------------------------------------------------------------------
%	WORK EXPERIENCE SECTION
%----------------------------------------------------------------------------------------

\begin{rSection}{Research Experience}

	\begin{rSubsection}{Stanford Institute for Economic Policy Research (SIEPR), Stanford University}{\textit{July '25 -- present}}{Predoctoral Research Fellow}{Palo Alto, CA}
            \item Predoctoral research fellow to Professors \href{https://ranabr.people.stanford.edu/}{Ran Abramitzky} and \href{https://www.beckylesteracctg.com/}{Rebecca Lester}.
	\end{rSubsection}

%------------------------------------------------
    
	\begin{rSubsection}{Economic Policy Studies, American Enterprise Institute (AEI)}{\textit{July '24 -- May '25}}{Research Assistant}{Washington, D.C.}
            \item Research assistant to PhD economists \href{https://www.aei.org/profile/stan-veuger/}{Stan Veuger}, \href{https://www.aei.org/profile/paul-h-kupiec/}{Paul Kupiec}, and \href{https://www.aei.org/profile/vincent-h-smith/}{Vincent Smith}.
	\end{rSubsection}

%------------------------------------------------

	\begin{rSubsection}{University of Hawai'i Economic Research Organization (UHERO)}{\textit{Nov. '23 -- June '24}}{Research Assistant}{Honolulu, HI}
        \item Research assistant to PhD economists \href{https://uhero.hawaii.edu/people/dylan-moore/}{Dylan Moore}, \href{https://uhero.hawaii.edu/people/peter-fuleky/}{Peter Fuleky}, and \href{https://uhero.hawaii.edu/people/rachel-inafuku/}{Rachel Inafuku}.
		\item Contributed to UHERO economists' research, including quarterly economic forecasts and academic research.
		\item Participated in data collection, analysis, and interpretation using statistical software.
	\end{rSubsection}

%------------------------------------------------

\newpage

	\begin{rSubsection}{Grassroot Institute of Hawaii}{\textit{July '21 -- Dec. '23}}{Research Associate}{Honolulu, HI}
		\item Employed econometric tools via R to produce  research relevant to Hawaii public policy.
		\item Research contributions cited in publications such as Wall Street Journal and Honolulu Star-Advertiser.
		\item Engaged in policy analysis and development, focusing on housing affordability.
		\item Presented research to state and local lawmakers and at public policy forums.
	\end{rSubsection}
 
 %------------------------------------------------

	\begin{rSubsection}{Center for Entrepreneurship \& Economic Education}{\textit{Aug. -- Nov. '23}}{Research Fellow}{Honolulu, HI}
		\item Research assistant to PhD economist \href{https://www.linkedin.com/in/gerard-dericks-569909a/}{Gerard Dericks}.
		\item Coauthoring an in-progress academic paper on housing prices and supply restrictions.
		\item Conducted comprehensive literature reviews and data analysis using R and Excel.
		\item Contributed to the development of empirical strategy and formal research design.
	\end{rSubsection}

 %------------------------------------------------

	\begin{rSubsection}{Urban Land Institute}{\textit{Summer '21 \& '22}}{Institute Scholar}{Honolulu, HI}
		\item Collaborated with fellow participants and experts on urban development projects focusing on workforce housing in Oahu.
		\item Engaged in community outreach and stakeholder meetings to gather input on housing projects.
		\item Presented proposals in front of local lawmakers and industry executives.
	\end{rSubsection}

\end{rSection}

\begin{rSection}{Other Experience}

	\begin{rSubsection}{TBSE Economics Bridge Program}{\textit{Jan. - June '24}}{Ambassador}{Honolulu, HI}
            \item Aided and facilitated the transition of students from community college and underrepresented backgrounds to UH M\=anoa's economics program.
	\end{rSubsection}
 \end{rSection}

\begin{rSection}{Policy Reports}
    \textit{\textbf{Wealth by Association? How Social Networks Drive Inequality in Hawaii}}. (UHERO Brief) \hfill \textit{April '24} \smallskip\\
\end{rSection}

\begin{rSection}{Works in progress}
\textit{\textbf{Population Drops in Attractive Areas}} \\ (with Ed Glaeser, Ferdinando Monte, and Stan Veuger) \hfill \textit{In progress} \smallskip\\
\textit{\textbf{Impacts of Emergency Food Aid on Recipient Countries' Economic Outcomes}} \\ (with Phil Hoxie and Vincent Smith) \hfill \textit{In progress} \smallskip\\
\textit{\textbf{What Can ‘Longest-Run’ House Price Indices Tell Us About the Current Housing Crisis?}} \\ (with Gerard Dericks) \hfill \textit{In progress} \smallskip
\end{rSection}

\begin{rSection}{Conference Presentations (Invited \& Attended)}
    \textbf{\textit{What can ‘longest-run’ house price indices tell us about the current housing crisis?}} \\
    \textit{People, Planet and Prosperity Conference}, Chaminade University, Honolulu, HI \hfill \textit{August '22}
    \smallskip

    \textbf{\textit{Taxation-Induced Tenancy: A Case Study of Washington D.C.'s Vacancy Tax}} \\
    \textit{Undergraduate Showcase} (Presented), University of Hawaii at Mānoa, Honolulu, HI \hfill \textit{December '23} \\
    \textit{National Conference on Undergraduate Research} (Invited), The Council for Undergraduate Research, Long Beach, CA \hfill \textit{April '24} \\
    \textit{North American Meeting of the Urban Economics Association} (Presented), Washington, DC \hfill \textit{September '24} \\
    \textit{National Tax Association Annual Conference on Taxation} (Invited), Detroit, MI \hfill \textit{November '24}
\end{rSection}

\newpage

\begin{rSection}{References}
\begin{tabular}{ @{} l @{\hspace{2ex}} l @{\hspace{6ex}} l @{\hspace{2ex}} l }
  \textbf{Name} & \href{https://www2.hawaii.edu/~fuleky/}{Peter Fuleky, PhD} & \textbf{Name} & \href{https://www.dylantmoore.com/}{Dylan Moore, PhD} \\
  \textbf{Title} & Professor of Economics & \textbf{Title} & Assistant Professor of Economics \\
  \textbf{Institution(s)} & UHERO, UHM & \textbf{Institution(s)} & UHERO, UHM \\
  \textbf{Email} & fuleky@hawaii.edu & \textbf{Email} & dtmoore@hawaii.edu \\
  \textbf{Phone} & (808) 956-7840 & \textbf{Phone} & (734) 881-0673 \\
  \\[\medskipamount]
  \textbf{Name} & \href{https://www.linkedin.com/in/gerard-dericks-569909a/}{Gerard Dericks, PhD} & \textbf{Name} & \href{https://sites.google.com/view/rachel-inafuku-phd/}{Rachel Inafuku, PhD} \\
  \textbf{Title} & Professor \& Director & \textbf{Title} & Research Economist \\
  \textbf{Institution(s)} & CEEE, Hawaii Pacific University & \textbf{Institution(s)} & UHERO \\
  \textbf{Email} & gdericks@hpu.edu & \textbf{Email} & rinafuku@hawaii.edu \\
  \textbf{Phone} & (808) 807-7557 & \textbf{Phone} & (808) 381-5286 \\
\end{tabular}
\end{rSection}

%----------------------------------------------------------------------------------------
%	EXAMPLE SECTION
%----------------------------------------------------------------------------------------

%\begin{rSection}{Section Name}

	%Section content\ldots

%\end{rSection}

%----------------------------------------------------------------------------------------

\end{document}
